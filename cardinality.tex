
\documentclass[]{article}
\voffset=-1.5cm
\oddsidemargin=0.0cm
\textwidth = 470pt
\usepackage[utf8]{inputenc}
\usepackage[english]{babel}
\usepackage{framed}

\usepackage{multicol}
\usepackage{amsmath}
\usepackage{amssymb}
\usepackage{enumerate}
\usepackage{multicol}

\subsection{Cardinality of a set}
A set in called finite when it contains a finite number of elements, and 
otherwise it is called infinite.

The number of distinct elements in a finite set is called its cardinality.


\begin{itemize}
\item The cardinality of a set A is the number of elements in A, which is written as $|A|$.
\item An element of a set is any one of the \textbf{distinct} objects that make up that set.

\end{itemize}

%---------------------------------------------------- %

\begin{itemize}

\item Note that this vertical-bar notation looks the same as absolute value notation, 
but the meaning of cardinality is different from absolute value.

\item In particular, absolute value operates on numbers (e.g., $|-4| = 4$) 
while cardinality operates on sets (e.g., $|\{-4\}| = 1$).
\end{itemize}

\textbf{Cardinality}

The cardinality $\|X\|$ of X is, roughly speaking, its size. The empty set has size 0, while B = $\{red, green, blue\}$ has size 3. 

This method of expressing cardinality works for finite sets but is not helpful for infinite ones. A more useful notion is that of two sets having the same size: if a direct one-to-one correspondence can be found between the elements of X and those of Y, they have the same size. 

Consider the sets, both infinite, of positive integers $\{1,2,3,4, \ldots\}$ and of even positive integers $\{2,4,6,8, \ldots\}$. One is a subset of the other. Nevertheless they have the same cardinality, as is shown by the correspondence mapping n in the former set to 2n in the latter. We can express this by writing $\|X\|$ = $\|Y\|$, or by saying that X and Y are "equinumerous".
%---------------------------------------------------- %

\textbf{Examples}
\begin{itemize}
\item[(i)] $|\{2,6,7\}|$
\item[(ii)] $|\{5,6,5,2,2,6,5,1,1,1\}|$
\item[(iii)] $|\{ \;\}| = 0$. %The empty set has no elements.
\item[(iv)] $|\{\{1,2\},\{3,4\}\}| = 2$. 
%In this case the two elements of $\{\{1,2\},\{3,4\}\}$ are themselves sets: $\{1,2\}$ and $\{3,4\}$.
\end{itemize}


%---------------------------------------------------- %
%---------------------------------------------------- %

\textbf{Examples}
\begin{itemize}
\item[(i)] $|\{2,6,7\}| $
\vspace{2cm}
\item[(ii)] $|\{5,6,5,2,2,6,5,1,1,1\}| $

\item[(i)] $|\{2,6,7\}| = 3 $
\vspace{2cm}
\item[(ii)] $|\{5,6,5,2,2,6,5,1,1,1\}| = |\{1,2,5,6\}| $
\end{itemize}


%---------------------------------------------------- %
%---------------------------------------------------- %

%% - \frametitle{Examples of cardinality}
\LARGE
\vspace{-1cm}
\textbf{Examples}
\begin{itemize}
\item[(i)] $|\{2,6,7\}| $
\vspace{2cm}
\item[(ii)] $|\{5,6,5,2,2,6,5,1,1,1\}| = |\{1,2,5,6\}| $
\end{itemize}


%---------------------------------------------------- %
%---------------------------------------------------- %



\textbf{Examples}
\begin{itemize}
\item[(i)] $|\{2,6,7\}| = 3$
\vspace{2cm}
\item[(ii)] $|\{5,6,5,2,2,6,5,1,1,1\}| = |\{1,2,5,6\}| = 4$

\item[(iii)] $|\{ \; \}| $
\vspace{2cm}
\item[(iv)] $|\{\{1,2\},\{3,4\}\}| $.
\end{itemize}


%---------------------------------------------------- %
%---------------------------------------------------- %



\textbf{Examples}
\begin{itemize}
\item[(iii)] $|\{ \; \}| = 0$. \\ The empty set has no elements.
\vspace{0.4cm} 
\item[(iv)] $|\{\{1,2\},\{3,4\}\}| = 2$. \\ \vspace{0.4cm} In this case the two elements of $\{\{1,2\},\{3,4\}\}$ are themselves sets: $\{1,2\}$ and $\{3,4\}$.
\end{itemize}




\subsection{Cardinality of a Set}
Cardinality of a set S, denoted by $|S|$, is the number of elements of the set. The number is also referred as the cardinal number. If a set has an infinite number of elements, its cardinality is $\infty$.

Example − |{1,4,3,5}|=4,|{1,2,3,4,5,…}|=∞

If there are two sets X and Y,

|X|=|Y||X|=|Y| denotes two sets X and Y having same cardinality. It occurs when the number of elements in X is exactly equal to the number of elements in Y. In this case, there exists a bijective function ‘f’ from X to Y.

$|X|\leq|Y|$$ denotes that set X’s cardinality is less than or equal to set Y’s cardinality. It occurs when number of elements in X is less than or equal to that of Y. Here, there exists an injective function ‘f’ from X to Y.

$|X|<|Y|$$ denotes that set X’s cardinality is less than set Y’s cardinality. It occurs when number of elements in X is less than that of Y. Here, the function ‘f’ from X to Y is injective function but not bijective.

If $|X|\leq|Y|$ and $|X|≥|Y|$ then $|X|=|Y|$. The sets X and Y are commonly referred as equivalent sets.

%--------------------------%
\end{document}
