
%======================================================================== %
\section{Set operations}

Suppose X and Y are sets. Various operations allow us to build new sets from them.

\begin{description}
\item[Union]
The union of X and Y, written $X\cup Y$, contains all the elements in X and all those in Y. Thus $A \cup B = \{1, 2, 3, red, green, blue\}$. As A is a subset of E, the set $A \cup E$ is just E.

\item[Intersection]

The intersection of X and Y, written X∩Y, contains all the elements that are common to both X and Y. Thus {1,2,3,red,green,blue} ∩ {2,4,6,8,10} = {2}.

\item[Set difference]

The difference X minus Y, written X−Y or $X\|Y$, contains all those elements in X that are not also in Y. For example, E−A contains all integers greater than 3. A−B is just A; red, green and blue were not elements of A, so no difference is made by excluding them.
\end{description}


%---------------------------------%

\subsection{Important Operations in Set Theory}

\begin{itemize}
\item Union ($\cup$) - also known as the OR operator. A union signifies a bringing together. The union of the sets A and B consists of the elements that are in either A or B.
\item Intersection ($\cap$) - also known as the AND operator. An intersection is where two things meet. The intersection of the sets A and B consists of the elements that in both A and B.
\item Complement ($^{c}$) - The complement of the set A consists of all of the elements in the universal set that are not elements of A.
\end{itemize}



%================================= %
\subsection*{Example 1: }
\begin{verbatim}
If $U = {1, 2, 3, 4, 5, 6, 7, 8, 9, 10}$, \\

$A = {2, 4, 6, 8, 10}$, \\
$B = (1, 3, 6, 7, 8}$ \\
$C = {3, 7}$, \\

find $A \cap B$, $A \cup C$, $B \cap A^{c}$, $B \cap C^{c}$
\end{verbatim}
\begin{verbatim}
Solution: 

$U = {1, 2, 3, 4, 5, 6, 7, 8, 9, 10}$\\
$A = {2, 4, 6, 8, 10}$\\
%$B = (1, 3, 6, 7, 8} $\\
$C = {3, 7}$\\

%A n B = {6, 8}
%A \cap C = {2, 3, 4, 6, 7, 8, 10}
%B n A^C = {1, 3, 7}
%B n C^C = {1, 6, 8}
\end{verbatim}
\begin{verbatim}

%> A
%[1]  1  8  9 10
%> B
%[1] 4 7 9
%> C
%[1] 1 2 3 4 9
%> D
%[1] 2 5 6
\end{verbatim}






%--------------------------------------%

\subsection*{Worked Example}

Suppose that the Universal Set $\mathcal{U}$ is the set of integers from 1 to 9.

\[ \mathcal{U} = \{1,2,3,4,5,6,7,8,9\}, \]

and that the set $\mathcal{A}$ contains the prime numbers between 1 to 9 inclusive.


\[ \mathcal{A} = \{1,2,3,5,7\}, \]


and that the set $\mathcal{B}$ contains the even numbers between 1 to 9 inclusive.


\[ \mathcal{B} = \{2,4,6,8\}. \]


%--------------------------------------------------------%

\subsubsection*{Complements}

\begin{itemize}


\item The Complements of A and B are the elements of the universal set not contained in A and B.


\item The complements are denoted $\mathcal{A}^{\prime}$ and $\mathcal{B}^{\prime}$

\[ \mathcal{A}^{\prime} = \{4,6,8,9\}, \]


\[ \mathcal{B}^{\prime} = \{1,3,5,7,9\}, \]


\end{itemize}



%--------------------------------------------------------%



\subsubsection*{Intersection}
\begin{itemize}


\item Intersection of two sets describes the elements that are members of both the specified Sets


\item The intersection is denoted $\mathcal{A\cap B}$ 
\[ \mathcal{A\cap B} = \{2\}\]


\item only one element is a member of both A and B.

\end{itemize}

%--------------------------------------------------------%


\subsubsection*{Set Difference}

\begin{itemize}


\item The Set Difference of A with regard to B are list of elements of A not contained by B.


\item The complements are denoted $\mathcal{A-B}$ and $\mathcal{B-A}$

\[ \mathcal{A-B} = \{1,3,5,7\}, \]


\[ \mathcal{B-A} = \{4,6,8\}, \]

\end{itemize}

\subsection*{symbols}
$\varnothing$,
$\forall$,
$\in$,
$\notin$,
$\cup$

%----------------------------------------------------------- %

\newpage


%--------------------------------------------------------%

\subsubsection*{Relative Difference}
\begin{itemize}
\item $ A \otimes B$
\end{itemize}

%--------------------------------------------------------%
Set Operations
Consider the universal set U such that U={1,2,3,4,5,6,7,8,9}  and the sets A={2,5,7,9}  and  B={2,4,6,8,9} 
Perform the following binary operations

(a) A-B
(b) A ⊕B
(c) A∩B
(d) A∪B
(e) A'∪B'
(f) A'∩B'
