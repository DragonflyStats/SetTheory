
%--------------------------------------%

\subsection*{Worked Example}

Suppose that the Universal Set $\mathcal{U}$ is the set of integers from 1 to 9.

\[ \mathcal{U} = \{1,2,3,4,5,6,7,8,9\}, \]

and that the set $\mathcal{A}$ contains the prime numbers between 1 to 9 inclusive.


\[ \mathcal{A} = \{1,2,3,5,7\}, \]


and that the set $\mathcal{B}$ contains the even numbers between 1 to 9 inclusive.


\[ \mathcal{B} = \{2,4,6,8\}. \]


%--------------------------------------------------------%

\subsubsection*{Complements}

\begin{itemize}


\item The Complements of A and B are the elements of the universal set not contained in A and B.


\item The complements are denoted $\mathcal{A}^{\prime}$ and $\mathcal{B}^{\prime}$

\[ \mathcal{A}^{\prime} = \{4,6,8,9\}, \]


\[ \mathcal{B}^{\prime} = \{1,3,5,7,9\}, \]


\end{itemize}



%--------------------------------------------------------%



\subsubsection*{Intersection}
\begin{itemize}


\item Intersection of two sets describes the elements that are members of both the specified Sets


\item The intersection is denoted $\mathcal{A\cap B}$ 
\[ \mathcal{A\cap B} = \{2\}\]


\item only one element is a member of both A and B.

\end{itemize}

%--------------------------------------------------------%


\subsubsection*{Set Difference}

\begin{itemize}


\item The Set Difference of A with regard to B are list of elements of A not contained by B.


\item The complements are denoted $\mathcal{A-B}$ and $\mathcal{B-A}$

\[ \mathcal{A-B} = \{1,3,5,7\}, \]


\[ \mathcal{B-A} = \{4,6,8\}, \]

\end{itemize}

\subsection*{symbols}
$\varnothing$,
$\forall$,
$\in$,
$\notin$,
$\cup$

%----------------------------------------------------------- %

\newpage


%--------------------------------------------------------%

\subsubsection*{Relative Difference}
\begin{itemize}
\item $ A \otimes B$
\end{itemize}

%--------------------------------------------------------%
