
\documentclass[11pt,a4paper,titlepage,oneside,openany]{article}

\pagestyle{plain}
%\renewcommand{\baselinestretch}{1.7}

\usepackage{setspace}
%\singlespacing
\onehalfspacing
%\doublespacing
%\setstretch{1.1}

\usepackage{amsmath}
\usepackage{amssymb}
\usepackage{amsthm}
\usepackage{framed}
\usepackage{multicol}

\usepackage[margin=3cm]{geometry}
\usepackage{graphicx,psfrag}%\usepackage{hyperref}
\usepackage[small]{caption}
\usepackage{subfig}

\usepackage{algorithm}
\usepackage{algorithmic}
\newcommand{\theHalgorithm}{\arabic{algorithm}}

\usepackage{varioref} %NB: FIGURE LABELS MUST ALWAYS COME DIRECTLY AFTER CAPTION!!!
%\newcommand{\vref}{\ref}

\usepackage{index}
\makeindex
\newindex{sym}{adx}{and}{Symbol Index}
%\newcommand{\symindex}{\index[sym]}
%\newcommand{\symindex}[1]{\index[sym]{#1}\hfill}
\newcommand{\symindex}[1]{\index[sym]{#1}}

%\usepackage[breaklinks,dvips]{hyperref}%Always put after varioref, or you'll get nested section headings
%Make sure this is after index package too!
%\hypersetup{colorlinks=false,breaklinks=true}
%\hypersetup{colorlinks=false,breaklinks=true,pdfborder={0 0 0.15}}


%\usepackage{breakurl}

\graphicspath{{./images/}}

\usepackage[subfigure]{tocloft}%For table of contents
\setlength{\cftfignumwidth}{3em}

\input{longdiv}
\usepackage{wrapfig}


%\usepackage{index}
%\makeindex
%\usepackage{makeidx}

%\usepackage{lscape}
\usepackage{pdflscape}
\usepackage{multicol}

\usepackage[utf8]{inputenc}

%\usepackage{fullpage}

%Compulsory packages for the PhD in UL:
%\usepackage{UL Thesis}
\usepackage{natbib}

%\numberwithin{equation}{section}
\numberwithin{equation}{section}
\numberwithin{algorithm}{section}
\numberwithin{figure}{section}
\numberwithin{table}{section}
%\newcommand{\vec}[1]{\ensuremath{\math{#1}}}

%\linespread{1.6} %for double line spacing

\usepackage{afterpage}%fingers crossed

\newtheorem{thm}{Theorem}[section]
\newtheorem{defin}{Definition}[section]
\newtheorem{cor}[thm]{Corollary}
\newtheorem{lem}[thm]{Lemma}

%\newcommand{\dbar}{{\mkern+3mu\mathchar'26\mkern-12mu d}}
\newcommand{\dbar}{{\mkern+3mu\mathchar'26\mkern-12mud}}

\newcommand{\bbSigma}{{\mkern+8mu\mathsf{\Sigma}\mkern-9mu{\Sigma}}}
\newcommand{\thrfor}{{\Rightarrow}}

\newcommand{\mb}{\mathbb}
\newcommand{\bx}{\vec{x}}
\newcommand{\bxi}{\boldsymbol{\xi}}
\newcommand{\bdeta}{\boldsymbol{\eta}}
\newcommand{\bldeta}{\boldsymbol{\eta}}
\newcommand{\bgamma}{\boldsymbol{\gamma}}
\newcommand{\bTheta}{\boldsymbol{\Theta}}
\newcommand{\balpha}{\boldsymbol{\alpha}}
\newcommand{\bmu}{\boldsymbol{\mu}}
\newcommand{\bnu}{\boldsymbol{\nu}}
\newcommand{\bsigma}{\boldsymbol{\sigma}}
\newcommand{\bdiff}{\boldsymbol{\partial}}

\newcommand{\tomega}{\widetilde{\omega}}
\newcommand{\tbdeta}{\widetilde{\bdeta}}
\newcommand{\tbxi}{\widetilde{\bxi}}



\newcommand{\wv}{\vec{w}}

\newcommand{\ie}{i.e. }
\newcommand{\eg}{e.g. }
\newcommand{\etc}{etc}

\newcommand{\viceversa}{vice versa}
\newcommand{\FT}{\mathcal{F}}
\newcommand{\IFT}{\mathcal{F}^{-1}}
%\renewcommand{\vec}[1]{\boldsymbol{#1}}
\renewcommand{\vec}[1]{\mathbf{#1}}
\newcommand{\anged}[1]{\langle #1 \rangle}
\newcommand{\grv}[1]{\grave{#1}}
\newcommand{\asinh}{\sinh^{-1}}

\newcommand{\sgn}{\text{sgn}}
\newcommand{\morm}[1]{|\det #1 |}

\newcommand{\galpha}{\grv{\alpha}}
\newcommand{\gbeta}{\grv{\beta}}
%\newcommand{\rnlessO}{\mb{R}^n \setminus \vec{0}}
\usepackage{listings}

\interfootnotelinepenalty=10000

\newcommand{\sectionline}{%
  \nointerlineskip \vspace{\baselineskip}%
  \hspace{\fill}\rule{0.5\linewidth}{.7pt}\hspace{\fill}%
  \par\nointerlineskip \vspace{\baselineskip}
}

\renewcommand{\labelenumii}{\roman{enumii})}

\begin{document}

%--------------------------------------------------------%
\subsubsection*{Complements}
\begin{itemize}

\item The Complements of A and B are the elements of the universal set not contained in A and B.

\item The complements are denoted $\mathcal{A}^{c}$ and $\mathcal{B}^{c}$
\[ \mathcal{A}^{c} = \{4,6,8,9\}, \]
\[ \mathcal{B}^{c} = \{1,3,5,7,9\}, \]

\end{itemize}


%--------------------------------------------------------%

\subsubsection*{Intersection}
\begin{itemize}

\item Intersection of two sets describes the elements that are members of both the specified Sets

\item The intersection is denoted $\mathcal{A\cap B}$
\[ \mathcal{A\cap B} = \{2\}\]

\item only one element is a member of both A and B.
\end{itemize}
%--------------------------------------------------------%

\subsubsection*{Set Difference}
\begin{itemize}

\item The Set Difference of A with regard to B are list of elements of A not contained by B.

\item The complements are denoted $\mathcal{A-B}$ and $\mathcal{B-A}$
\[ \mathcal{A-B} = \{1,3,5,7\}, \]

\[ \mathcal{B-A} = \{4,6,8\}, \]
\end{itemize}

\subsection*{Union and intersection of sets}

\begin{itemize}
\item The \textbf{union} of two sets A and B is a set containing all the elements in
either A or B (or both)
i.e.
\[A \cup B = {x / x \in A \mbox{ or } x \in B}.\]
\item The \textbf{intersection} of two sets A and B is a set containing all the elements
that are both in A and B
i.e.
\[A \cap B = {x / x \in A \mbox{ and }x \in B}\].

\item If sets A and B have no elements in common, i.e. $A \cap B = \emptyset$,then A and B
are termed \textbf{disjoint sets}.
\end{itemize}

%------------------------------------------------------%

\section*{Complement of a Set}
%(2.3.1)
Consider the universal set $U$ such that
\[U=\{2,4,6,8,10,12,15\} \]
For each of the sets $A$,$B$,$C$ and $D$, specify the complement sets.
\begin{center}
\begin{tabular}{|c|c|}
\hline
Set & Complement\\
\hline $A=\{4,6,12,15\}$ &
$A^{c}=\{2,8,10\}$ \\ \hline $B=\{4,8,10,15\}$ & \\ \hline
$C=\{2,6,12,15\}$ & \\ \hline $D=\{8,10,15\}$ & \\ \hline

\end{tabular}
\end{center}

\subsection{Set operations}

Suppose X and Y are sets. Various operations allow us to build new sets from them.

\begin{description}
\item[Union]
The union of X and Y, written $X\cup Y$, contains all the elements in X and all those in Y. Thus $A \cup B = \{1, 2, 3, red, green, blue\}$. 
As A is a subset of E, the set $A \cup E$ is just E.

\item[Intersection]

The intersection of X and Y, written $X \cap Y$, contains all the elements that are common to both X and Y. Thus ${1,2,3,red,green,blue} \cap {2,4,6,8,10} = {2}$.

\item[Set difference]

The difference X minus Y, written X-Y or $X\|Y$, contains all those elements in X that are not also in Y. For example, E-A contains all integers greater than 3. 
A-B is just A; red, green and blue were not elements of A, so no difference is made by excluding them.
\end{description}

%---------------------------------%

\subsection{Important Operations in Set Theory}

\begin{itemize}
\item Union ($\cup$) - also known as the OR operator. A union signifies a bringing together. The union of the sets A and B consists of the elements that are in either A or B.
\item Intersection ($\cap$) - also known as the AND operator. An intersection is where two things meet. The intersection of the sets A and B consists of the elements that in both A and B.
\item Complement ($^{c}$) - The complement of the set A consists of all of the elements in the universal set that are not elements of A.
\end{itemize}

%---------------------------------------%
\subsection*{Dice Rolls}
Consider rolls of a die. What is the universal set?

\[ \mathcal{U} = \{1,2,3,4,5,6\} \]

%--------------------------------------%
\subsection*{Worked Example}

Suppose that the Universal Set $\mathcal{U}$ is the set of integers from 1 to 9.
\[ \mathcal{U} = \{1,2,3,4,5,6,7,8,9\}, \]

and that the set $\mathcal{A}$ contains the prime numbers between 1 to 9 inclusive.

\[ \mathcal{A} = \{1,2,3,5,7\}, \]

and that the set $\mathcal{B}$ contains the even numbers between 1 to 9 inclusive.

\[ \mathcal{B} = \{2,4,6,8\}. \]

%--------------------------------------------------------%
\subsubsection*{Complements}
\begin{itemize}

\item The Complements of A and B are the elements of the universal set not contained in A and B.

\item The complements are denoted $\mathcal{A}^{\prime}$ and $\mathcal{B}^{\prime}$
\[ \mathcal{A}^{\prime} = \{4,6,8,9\}, \]
\[ \mathcal{B}^{\prime} = \{1,3,5,7,9\}, \]

\end{itemize}


\subsubsection*{Intersection}
\begin{itemize}

\item Intersection of two sets describes the elements that are members of both the specified Sets

\item The intersection is denoted $\mathcal{A\cap B}$ 
\[ \mathcal{A\cap B} = \{2\}\]

\item only one element is a member of both A and B.
\end{itemize}
%--------------------------------------------------------%

\subsubsection*{Set Difference}
\begin{itemize}

\item The Set Difference of A with regard to B are list of elements of A not contained by B.

\item The complements are denoted $\mathcal{A-B}$ and $\mathcal{B-A}$
\[ \mathcal{A-B} = \{1,3,5,7\}, \]

\[ \mathcal{B-A} = \{4,6,8\}, \]
\end{itemize}
\subsection*{symbols}
$\varnothing$,
$\forall$,
$\in$,
$\notin$,
$\cup$
%----------------------------------------------------------- %
\newpage

\section{Set Operations}
Set Operations include Set Union, Set Intersection, Set Difference, Complement of Set, and Cartesian Product.

\subsection{Set Union}
The union of sets A and B (denoted by $A \cup B$) is the set of elements which are in A, in B, or in both A and B. Hence, $A \cup B={x|x \in \mbox{ A OR x } \in B}$.

Example - $If A={10,11,12,13}$ and $B = {13,14,15}$, then $A \cup B={10,11,12,13,14,15}$. (The common element occurs only once)

%=========================================%
\subsection{Set Intersection}
The intersection of sets A and B (denoted by $AnB$) is the set of elements which are in both A and B. Hence, $A \cap B={x|x \in A AND x \in B}$.

Example - If $A={11,12,13}$ and $B={13,14,15}$, then $A \cap B={13}$.
%=========================================%
\subsection{Set Difference/ Relative Complement}
The set difference of sets A and B (denoted by $A–B$) is the set of elements which are only in A but not in B. Hence, $A-B={x|x \in A AND x \notin B}$.
Example: If $A={10,11,12,13}$ and $B={13,14,15}$, then $(A-B)={10,11,12}$ and $(B-A)={14,15}$. Here, we can see (A-B)?(B-A)(A-B)?(B-A) Set Difference
%=========================================%
\subsection{Complement of a Set}
The complement of a set A (denoted by $A^{c}$) is the set of elements which are not in set A. Hence, $A^{c}={x|x \notin A}$.
More specifically, $A^{c}=(U-A)$ where UU is a universal set which contains all objects.

\subsection{Example} - If A={x|x belongstosetofoddintegers}A={x|x belongstosetofoddintegers} then $A^{c}={y|y \mbox{does not belong toset ofodd integers}}$

Complement Set
%======================================================================================= %
\newpage
\section*{Set Operations}
\begin{itemize}
	\item Union ($\cup$) - also known as the \textbf{OR} operator. A union signifies a bringing together. The union of the sets A and B consists of the elements that are in either A or B.
	\item Intersection ($\cap$) - also known as the \textbf{AND} operator. An intersection is where two things meet. The intersection of the sets A and B consists of the elements that in both A and B.
	\item Complement ($A^{c}$ or $A^{c}$) - The complement of the set A consists of all of the elements in the universal set that are not elements of A.
\end{itemize}
\subsection*{Exercise}
Consider the universal set $U$ such that
\[U=\{1,2,3,4,5,6,7,8,9\} \]
and the sets
\[A=\{2,5,7,9\} \]
\[B=\{2,4,6,8,9\} \]
\begin{multicols}{2}
\begin{itemize}
	\item[(a)] $A-B$
	\item[(b)] $A \otimes B$
	\item[(c)] $A \cap B$
	\item[(d)] $A \cup B$
	\item[(e)] $A^{c} \cap B^{c}$
	\item[(f)] $A^{c} \cup B^{c}$
\end{itemize}
\end{multicols}
\newpage
\subsection*{Union and intersection of sets}
\begin{itemize}
\item The \textbf{union} of two sets A and B is a set containing all the elements in
either A or B (or both)
i.e.
\[A \cup B = {x / x \in A \mbox{ or } x \in B}.\]
\item The \textbf{intersection} of two sets A and B is a set containing all the elements
that are both in A and B
i.e.
\[A \cap B = {x / x \in A \mbox{ and }x \in B}\].
\item If sets A and B have no elements in common, i.e. $A \cap B = \emptyset$,then A and B
are termed \textbf{disjoint sets}.
\end{itemize}
\newpage
%--------------------------------------------------------%
\subsubsection*{Intersection}
\begin{itemize}
\item Intersection of two sets describes the elements that are members of both the specified Sets
\item The intersection is denoted $\mathcal{A\cap B}$
\[ \mathcal{A\cap B} = \{2\}\]
\item only one element is a member of both A and B.
\end{itemize}
%--------------------------------------------------------%
\subsubsection*{Set Difference}
\begin{itemize}
\item The Set Difference of A with regard to B are list of elements of A not contained by B.
\item The complements are denoted $\mathcal{A-B}$ and $\mathcal{B-A}$
\[ \mathcal{A-B} = \{1,3,5,7\}, \]
\[ \mathcal{B-A} = \{4,6,8\}, \]
\end{itemize}
\subsection*{symbols}
$\varnothing$,
$\forall$,
$\in$,
$\notin$,
$\cup$
%----------------------------------------------------------- %
\newpage
%--------------------------------------------------------%
\subsubsection*{Relative Difference}
\begin{itemize}
\item $ A \otimes B$
\end{itemize}
%--------------------------------------------------------%
\subsubsection*{Power Sets}
\begin{itemize}
\item Consider the set A where $ A = \{w,x,y,z\}$
\item There are 4 elements in set A.
\item The power set of A contains 16 element data sets.
\item \[  \mathcal{P}(A) = \{\{ x \}, \{ y \} \}  \]
\item (i.e. 1 null set, 4 single element sets, 6 two -elemnts sets, 4 three lement set and one 4- element set.)
\end{itemize}
\end{document}
%======================================================================== %
\section{Set operations}
Suppose X and Y are sets. Various operations allow us to build new sets from them.
\begin{description}
\item[Union]
The union of X and Y, written $X\cup Y$, contains all the elements in X and all those in Y. Thus $A \cup B = \{1, 2, 3, red, green, blue\}$. As A is a subset of E, the set $A \cup E$ is just E.
\item[Intersection]
The intersection of X and Y, written XnY, contains all the elements that are common to both X and Y. Thus {1,2,3,red,green,blue} n {2,4,6,8,10} = {2}.
\item[Set difference]
The difference X minus Y, written X-Y or $X\|Y$, contains all those elements in X that are not also in Y. For example, E-A contains all integers greater than 3. A-B is just A; red, green and blue were not elements of A, so no difference is made by excluding them.
\end{description}
%---------------------------------%
\subsection{Important Operations in Set Theory}
\begin{itemize}
\item Union ($\cup$) - also known as the OR operator. A union signifies a bringing together. The union of the sets A and B consists of the elements that are in either A or B.
\item Intersection ($\cap$) - also known as the AND operator. An intersection is where two things meet. The intersection of the sets A and B consists of the elements that in both A and B.
\item Complement ($^{c}$) - The complement of the set A consists of all of the elements in the universal set that are not elements of A.
\end{itemize}
%================================= %
\subsection*{Example 1: }
\begin{verbatim}
If $U = {1, 2, 3, 4, 5, 6, 7, 8, 9, 10}$, \\
$A = {2, 4, 6, 8, 10}$, \\
$B = (1, 3, 6, 7, 8}$ \\
$C = {3, 7}$, \\
find $A \cap B$, $A \cup C$, $B \cap A^{c}$, $B \cap C^{c}$
\end{verbatim}
\begin{verbatim}
Solution:
$U = {1, 2, 3, 4, 5, 6, 7, 8, 9, 10}$\\
$A = {2, 4, 6, 8, 10}$\\
%$B = (1, 3, 6, 7, 8} $\\
$C = {3, 7}$\\
%A n B = {6, 8}
%A \cap C = {2, 3, 4, 6, 7, 8, 10}
%B n A^C = {1, 3, 7}
%B n C^C = {1, 6, 8}
\end{verbatim}
\begin{verbatim}
%> A
%[1]  1  8  9 10
%> B
%[1] 4 7 9
%> C
%[1] 1 2 3 4 9
%> D
%[1] 2 5 6
\end{verbatim}
%--------------------------------------%
\subsection*{Worked Example}
Suppose that the Universal Set $\mathcal{U}$ is the set of integers from 1 to 9.
\[ \mathcal{U} = \{1,2,3,4,5,6,7,8,9\}, \]
and that the set $\mathcal{A}$ contains the prime numbers between 1 to 9 inclusive.
\[ \mathcal{A} = \{1,2,3,5,7\}, \]
and that the set $\mathcal{B}$ contains the even numbers between 1 to 9 inclusive.
\[ \mathcal{B} = \{2,4,6,8\}. \]
%--------------------------------------------------------%
\subsubsection*{Complements}
\begin{itemize}
\item The Complements of A and B are the elements of the universal set not contained in A and B.
\item The complements are denoted $\mathcal{A}^{c}$ and $\mathcal{B}^{c}$
\[ \mathcal{A}^{c} = \{4,6,8,9\}, \]
\[ \mathcal{B}^{c} = \{1,3,5,7,9\}, \]
\end{itemize}
%--------------------------------------------------------%
\subsubsection*{Intersection}
\begin{itemize}
\item Intersection of two sets describes the elements that are members of both the specified Sets
\item The intersection is denoted $\mathcal{A\cap B}$
\[ \mathcal{A\cap B} = \{2\}\]
\item only one element is a member of both A and B.
\end{itemize}
%--------------------------------------------------------%
\subsubsection*{Set Difference}
\begin{itemize}
\item The Set Difference of A with regard to B are list of elements of A not contained by B.
\item The complements are denoted $\mathcal{A-B}$ and $\mathcal{B-A}$
\[ \mathcal{A-B} = \{1,3,5,7\}, \]
\[ \mathcal{B-A} = \{4,6,8\}, \]
\end{itemize}
\subsection*{symbols}
$\varnothing$,
$\forall$,
$\in$,
$\notin$,
$\cup$
%----------------------------------------------------------- %
\newpage
%--------------------------------------------------------%
\subsubsection*{Relative Difference}
\begin{itemize}
\item $ A \otimes B$
\end{itemize}
%--------------------------------------------------------%
Set Operations
Consider the universal set U such that U={1,2,3,4,5,6,7,8,9}  and the sets A={2,5,7,9}  and  B={2,4,6,8,9}
Perform the following binary operations
\begin{enumerate}
\item A-B
\item  A ?B
\item  AnB
\item A?B
\item  A^{c}?B^{c}
\item A^{c}nB^{c}
\section{Differences and complements}
	\begin{itemize}
		\item If A and B are sets then the difference set A - B is the set of all elements
		of A which do not belong to B.
		\item If B is a sub-set of A, then A - B is sometimes called the complement of
		B in A. When A is the universal set one may simply refer to the
		complement of B to denote all things not in B. \item The complement of a set A
		is denoted as $A^c$ or $A^{c}$.
	\end{itemize}



\end{document}
