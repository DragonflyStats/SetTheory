
\documentclass[12pt]{article}
%\usepackage[final]{pdfpages}

\usepackage{graphicx}
\graphicspath{{/Users/kevinhayes/Documents/teaching/images/}}

\usepackage{tikz}
\usetikzlibrary{arrows}

\newcommand{\bbr}{\Bbb{R}}
\newcommand{\zn}{\Bbb{Z}^n}

%\usepackage{epsfig}
%\usepackage{subfigure}
\usepackage{amscd}
\usepackage{amssymb}
\usepackage{amsbsy}
\usepackage{amsthm}
\usepackage{natbib}
\usepackage{amsbsy}
\usepackage{enumerate}
\usepackage{amsmath}
\usepackage{eurosym}
%\usepackage{beamerarticle}
\usepackage{txfonts}
\usepackage{fancyvrb}
\usepackage{fancyhdr}
\usepackage{natbib}
\bibliographystyle{chicago}

\usepackage{vmargin}
% left top textwidth textheight headheight
% headsep footheight footskip
\setmargins{2.0cm}{2.5cm}{16 cm}{22cm}{0.5cm}{0cm}{1cm}{1cm}
\renewcommand{\baselinestretch}{1.3}


\pagenumbering{arabic}

\begin{document}


\section{Set Theory}
A set, in mathematics, is a collection of distinct entities, called its elements, considered as a whole. The early study of sets led to a family of paradoxes and apparent contradictions. It therefore became necessary to abandon "naïve" conceptions of sets, and a precise definition that avoids the paradoxes turns out to be a tricky matter. However, some unproblematic examples from naïve set theory will make the concept clearer. These examples will be used throughout this article:

\begin{itemize}
\item A = the set of the numbers 1, 2 and 3.
\item B = the set of primary light colours—red, green and blue.
\item C = the empty set (the set with no elements).
\item D = the set of all books in the British Library.
\item E = the set of all positive integers, 1, 2, 3, 4, and so on.
\end{itemize}

Note that the last of these sets is infinite.

A set is the collection of its elements considered as a single, abstract entity. Note that this is different from the elements themselves, and may have different properties. For example, the elements of D are flammable (they are books), but D itself is not flammable, since abstract objects cannot be burnt.


	
\section{Set Theory Introduction}


\begin{itemize}
\item A set is simply a collection of things or objects, of any kind. 
\item These objects
are called elements or members of the set. We refer to the set as an
entity in its own right and often denote it by A, B, C or D, etc.
\item 
If A is a set and x a member of the set, then we say $x \in A$ i.e. x ‘belongs to’
A. The symbol $\notin$ denotes the negation of $in$  i.e. x $\notin$ A means ‘x does not
belong to’ A.
\end{itemize}

\begin{itemize}
\item The elements of a set, and hence the set itself, are characterised by having
one or more properties that distinguish the elements of the set from those
not in the set. 
\item For example, if C is the set of non-negative real numbers, then we
might use the notation
\[C = \{x / x \mbox{ is a real number and} x \neq 0\}\]
\item We would verbalise this as \textit{the set of all x such that x is a real number and non-negative}.
\end{itemize}


\subsection{Set - Definition}
A set is an unordered collection of different elements. A set can be written explicitly by listing its elements using set bracket. If the order of the elements is changed or any element of a set is repeated, it does not make any changes in the set.

Some Example of Sets
\begin{itemize}
\item A set of all positive integers
\item A set of all the states in the USA
\item A set of all the rows in a dataframe
\item A set of all the lowercase letters of the greek alphabet
\end{itemize}
\section{Sets}
\smallskip   %GOOD

\begin{itemize}
\item A set is simply a collection of things or objects, of any kind. These objects
are called elements or members of the set. We refer to the set as an
entity in its own right and often denote it by A, B, C or D, etc.
\item If A is a set and x a member of the set, then we say $x \in A$ i.e. x ``belongs to``
A. 
\end{itemize}
\smallskip 
%---------------------------------------------------------------------------------------------------- %
% CIS 102
\section{Introduction} %GOOD - MERGE with above
\smallskip 
By a set we simply mean a collection or class of objects. The objects in the set are Called its
members or elements. Sets have become the basic language in which most results in mathematics
and computer science are expressed. 
In this chapter, we look at ways in which sets are specified,
how they may be represented and how they are combined to make other sets.
\smallskip 
%----------------------------------------------------------------------------------------------- %
\section{Introduction to Set Theory}
\begin{itemize}
\item The symbol $\notin$ denotes the negation of ? $in$.e. $x \notin A$ means ‘x does not
belong to’ A.
\item The elements of a set, and hence the set itself, are characterised by having
one or more properties that distinguish the elements of the set from those
not in the set.
\item For example, if C is the set of non-negative real numbers, then we
might use the notation
\[C = {x | \mbox{x is a real number and }x \neq 0}\]
i.e. the set of all x such that x is a real number and non-negative.
\end{itemize}


\begin{itemize}
\item A set is simply a collection of things or objects, of any kind. 
\item These objects
are called elements or members of the set. We refer to the set as an
entity in its own right and often denote it by A, B, C or D, etc.
\item 
If A is a set and x a member of the set, then we say $x \in A$ i.e. x ‘belongs to’
A. The symbol $\notin$ denotes the negation of $in$  i.e. x $\notin$ A means ‘x does not
belong to’ A.

\item The elements of a set, and hence the set itself, are characterised by having
one or more properties that distinguish the elements of the set from those
not in the set. 
\item For example, if C is the set of non-negative real numbers, then we
might use the notation
\[C = \{x / x \mbox{ is a real number and} x \neq 0\}\]
\item We would verbalise this as \textit{the set of all x such that x is a real number and non-negative}.
\end{itemize}
\end{document}
