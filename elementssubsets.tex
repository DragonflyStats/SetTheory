\documentclass[11pt,a4paper,titlepage,oneside,openany]{article}

\pagestyle{plain}
%\renewcommand{\baselinestretch}{1.7}

\usepackage{setspace}
%\singlespacing
\onehalfspacing
%\doublespacing
%\setstretch{1.1}

\usepackage{amsmath}
\usepackage{amssymb}
\usepackage{amsthm}
\usepackage{framed}
\usepackage{multicol}

\usepackage[margin=3cm]{geometry}
\usepackage{graphicx,psfrag}%\usepackage{hyperref}
\usepackage[small]{caption}
\usepackage{subfig}

\usepackage{algorithm}
\usepackage{algorithmic}
\newcommand{\theHalgorithm}{\arabic{algorithm}}

\usepackage{varioref} %NB: FIGURE LABELS MUST ALWAYS COME DIRECTLY AFTER CAPTION!!!
%\newcommand{\vref}{\ref}

\usepackage{index}
\makeindex
\newindex{sym}{adx}{and}{Symbol Index}
%\newcommand{\symindex}{\index[sym]}
%\newcommand{\symindex}[1]{\index[sym]{#1}\hfill}
\newcommand{\symindex}[1]{\index[sym]{#1}}

%\usepackage[breaklinks,dvips]{hyperref}%Always put after varioref, or you'll get nested section headings
%Make sure this is after index package too!
%\hypersetup{colorlinks=false,breaklinks=true}
%\hypersetup{colorlinks=false,breaklinks=true,pdfborder={0 0 0.15}}


%\usepackage{breakurl}

\graphicspath{{./images/}}

\usepackage[subfigure]{tocloft}%For table of contents
\setlength{\cftfignumwidth}{3em}

\input{longdiv}
\usepackage{wrapfig}


%\usepackage{index}
%\makeindex
%\usepackage{makeidx}

%\usepackage{lscape}
\usepackage{pdflscape}
\usepackage{multicol}

\usepackage[utf8]{inputenc}

%\usepackage{fullpage}

%Compulsory packages for the PhD in UL:
%\usepackage{UL Thesis}
\usepackage{natbib}

%\numberwithin{equation}{section}
\numberwithin{equation}{section}
\numberwithin{algorithm}{section}
\numberwithin{figure}{section}
\numberwithin{table}{section}
%\newcommand{\vec}[1]{\ensuremath{\math{#1}}}

%\linespread{1.6} %for double line spacing

\usepackage{afterpage}%fingers crossed

\newtheorem{thm}{Theorem}[section]
\newtheorem{defin}{Definition}[section]
\newtheorem{cor}[thm]{Corollary}
\newtheorem{lem}[thm]{Lemma}

%\newcommand{\dbar}{{\mkern+3mu\mathchar'26\mkern-12mu d}}
\newcommand{\dbar}{{\mkern+3mu\mathchar'26\mkern-12mud}}

\newcommand{\bbSigma}{{\mkern+8mu\mathsf{\Sigma}\mkern-9mu{\Sigma}}}
\newcommand{\thrfor}{{\Rightarrow}}

\newcommand{\mb}{\mathbb}
\newcommand{\bx}{\vec{x}}
\newcommand{\bxi}{\boldsymbol{\xi}}
\newcommand{\bdeta}{\boldsymbol{\eta}}
\newcommand{\bldeta}{\boldsymbol{\eta}}
\newcommand{\bgamma}{\boldsymbol{\gamma}}
\newcommand{\bTheta}{\boldsymbol{\Theta}}
\newcommand{\balpha}{\boldsymbol{\alpha}}
\newcommand{\bmu}{\boldsymbol{\mu}}
\newcommand{\bnu}{\boldsymbol{\nu}}
\newcommand{\bsigma}{\boldsymbol{\sigma}}
\newcommand{\bdiff}{\boldsymbol{\partial}}

\newcommand{\tomega}{\widetilde{\omega}}
\newcommand{\tbdeta}{\widetilde{\bdeta}}
\newcommand{\tbxi}{\widetilde{\bxi}}



\newcommand{\wv}{\vec{w}}

\newcommand{\ie}{i.e. }
\newcommand{\eg}{e.g. }
\newcommand{\etc}{etc}

\newcommand{\viceversa}{vice versa}
\newcommand{\FT}{\mathcal{F}}
\newcommand{\IFT}{\mathcal{F}^{-1}}
%\renewcommand{\vec}[1]{\boldsymbol{#1}}
\renewcommand{\vec}[1]{\mathbf{#1}}
\newcommand{\anged}[1]{\langle #1 \rangle}
\newcommand{\grv}[1]{\grave{#1}}
\newcommand{\asinh}{\sinh^{-1}}

\newcommand{\sgn}{\text{sgn}}
\newcommand{\morm}[1]{|\det #1 |}

\newcommand{\galpha}{\grv{\alpha}}
\newcommand{\gbeta}{\grv{\beta}}
%\newcommand{\rnlessO}{\mb{R}^n \setminus \vec{0}}
\usepackage{listings}

\interfootnotelinepenalty=10000

\newcommand{\sectionline}{%
  \nointerlineskip \vspace{\baselineskip}%
  \hspace{\fill}\rule{0.5\linewidth}{.7pt}\hspace{\fill}%
  \par\nointerlineskip \vspace{\baselineskip}
}

\renewcommand{\labelenumii}{\roman{enumii})}

\begin{document}

\subsection*{Elements of a Set}
\begin{itemize}
\item Sets are comprised of members, which are often called \textbf{elements}. 
\item If a particular value ($k$) is an element of set $A$, then we would write this as
\[k \in A \]

\item If a single value $k$ is not an element of set $A$, then we write
\[k \notin A \]
\end{itemize}

\subsection{Subsets}
Given two sets $A$ and $B$, the set $A$ is a \textbf{subset} of set $B$ if every element of $A$ is also an element of $B$. 


We write this mathematically as
\[A \subseteq B \]


\bigskip
Sets are denoted with curly braces, even if they contain only one element.


\subsection*{Subsets}\\
Suppose we have the set $A$ comprised of the following elements
\[ A =\{3,5,7,9\}\]
The value $5$ is an element of $A$
\[  5 \in A \]

The single element set $\{5\} $ is a subset of $A$.
\[ \{5\} \subseteq A\]

\section{Elements of a Set}

A set is defined completely by its elements. Formally, sets X and Y are the same set if they have the same elements; that is, if every element of X is also an element of Y, and vice versa. For example, suppose we define:

\[ F = \{x | \mbox{(x is an integer)} \mbox{ and } 0 < x < 4)\}  \]
%Then F=A. The definition also makes it clear that {1,2,3} = (3,2,1} = {1,1,2,3,2}. Note also that if X and Y are both empty, then they are equal, justifying referring to "the" empty set.

The equivalence of empty sets has a \textbf{\textit{metaphysical}} consequence for some theories of the metaphysics of properties that define the property of being x as simply the set of all x, then if the two properties are uninstantiated or coextensive they are equivalent - under this theory, because there are no unicorns and there are no pixies, the property of being a unicorn and being a pixie are the same - but if there were a unicorns and pixies, we could tell them apart. (See Universals for more on this.)

\subsection*{Elements and subsets}

The $\in$  sign indicates set membership. If x is an element (or "member") of a set X, we write $ \in X$; e.g. $3\in A$. (We may also say ``X contains x" and ``A contains 3")

A very important notion is that of a subset. X is a subset of Y, written $X \subseteq Y$ (sometimes simply as$X \subset Y$), if every element of X is also an element of Y. From before $C\subseteq A \subseteq E$.


\subsection*{Sets containing Sets}


Sets can of course be elements of other sets; for example we can form the set $G = \{A,B,C,D,E\}$, whose five elements are the sets we considered earlier. Then, for instance, $A\in G$. (Note that this is very different from saying $A\subseteq G$)

%The tilde (~) is as usual used for negation; e.g. $\tilde(A\in B)$.


\subsection{Equivalent Sets}
If both of the following two statements are \textbf{true}, 
\[\mbox{1)  } A \subseteq B \]
\[\mbox{2)  } B \subseteq A \]

then $A$ and $B$ are \textbf{equivalent sets}.



\subsection*{Non-Comparable Sets}\\
If both of the following two statements are \textbf{false}, 
\[\mbox{1)  } A \subseteq B \]
\[\mbox{2)  } B \subseteq A \]

then $A$ and $B$ are said to be said to be \textbf{noncomparable sets}.



%-------------------------------------%
\end{document}
