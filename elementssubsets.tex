
\documentclass[12pt]{article}
%\usepackage[final]{pdfpages}

\usepackage{graphicx}
\graphicspath{{/Users/kevinhayes/Documents/teaching/images/}}

\usepackage{tikz}
\usetikzlibrary{arrows}

\newcommand{\bbr}{\Bbb{R}}
\newcommand{\zn}{\Bbb{Z}^n}

%\usepackage{epsfig}
%\usepackage{subfigure}
\usepackage{amscd}
\usepackage{amssymb}
\usepackage{amsbsy}
\usepackage{amsthm}
\usepackage{natbib}
\usepackage{amsbsy}
\usepackage{enumerate}
\usepackage{amsmath}
\usepackage{eurosym}
%\usepackage{beamerarticle}
\usepackage{txfonts}
\usepackage{fancyvrb}
\usepackage{fancyhdr}
\usepackage{natbib}
\bibliographystyle{chicago}

\usepackage{vmargin}
% left top textwidth textheight headheight
% headsep footheight footskip
\setmargins{2.0cm}{2.5cm}{16 cm}{22cm}{0.5cm}{0cm}{1cm}{1cm}
\renewcommand{\baselinestretch}{1.3}


\pagenumbering{arabic}

\begin{document}

\subsection*{Elements of a Set}
\begin{itemize}
\item Sets are comprised of members, which are often called \textbf{elements}. 
\item If a particular value ($k$) is an element of set $A$, then we would write this as
\[k \in A \]

\item If a single value $k$ is not an element of set $A$, then we write
\[k \notin A \]
\end{itemize}

\subsection{Subsets}
Given two sets $A$ and $B$, the set $A$ is a \textbf{subset} of set $B$ if every element of $A$ is also an element of $B$. 


We write this mathematically as
\[A \subseteq B \]


\bigskip
Sets are denoted with curly braces, even if they contain only one element.


\subsection*{Subsets}
Suppose we have the set $A$ comprised of the following elements
\[ A =\{3,5,7,9\}\]
The value $5$ is an element of $A$
\[  5 \in A \]

The single element set $\{5\} $ is a subset of $A$.
\[ \{5\} \subseteq A\]

\section{Elements of a Set}

A set is defined completely by its elements. Formally, sets X and Y are the same set if they have the same elements; that is, if every element of X is also an element of Y, and vice versa. For example, suppose we define:

\[ F = \{x | \mbox{(x is an integer)} \mbox{ and } 0 < x < 4)\}  \]
%Then F=A. The definition also makes it clear that {1,2,3} = (3,2,1} = {1,1,2,3,2}. Note also that if X and Y are both empty, then they are equal, justifying referring to "the" empty set.

The equivalence of empty sets has a \textbf{\textit{metaphysical}} consequence for some theories of the metaphysics of properties that define the property of being x as simply the set of all x, then if the two properties are uninstantiated or coextensive they are equivalent - under this theory, because there are no unicorns and there are no pixies, the property of being a unicorn and being a pixie are the same - but if there were a unicorns and pixies, we could tell them apart. (See Universals for more on this.)

\subsection*{Elements and subsets}

The $\in$  sign indicates set membership. If x is an element (or "member") of a set X, we write $ \in X$; e.g. $3\in A$. (We may also say ``X contains x" and ``A contains 3")

A very important notion is that of a subset. X is a subset of Y, written $X \subseteq Y$ (sometimes simply as$X \subset Y$), if every element of X is also an element of Y. From before $C\subseteq A \subseteq E$.


\subsection*{Sets containing Sets}


Sets can of course be elements of other sets; for example we can form the set $G = \{A,B,C,D,E\}$, whose five elements are the sets we considered earlier. Then, for instance, $A\in G$. (Note that this is very different from saying $A\subseteq G$)

%The tilde (~) is as usual used for negation; e.g. $\tilde(A\in B)$.


\subsection{Equivalent Sets}
If both of the following two statements are \textbf{true}, 
\[\mbox{1)  } A \subseteq B \]
\[\mbox{2)  } B \subseteq A \]

then $A$ and $B$ are \textbf{equivalent sets}.



\subsection*{Non-Comparable Sets}
If both of the following two statements are \textbf{false}, 
\[\mbox{1)  } A \subseteq B \]
\[\mbox{2)  } B \subseteq A \]

then $A$ and $B$ are said to be said to be \textbf{noncomparable sets}.



%-------------------------------------%
\end{document}
