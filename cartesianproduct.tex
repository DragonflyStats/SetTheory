
\documentclass[12pt]{article}
%\usepackage[final]{pdfpages}

\usepackage{graphicx}
\graphicspath{{/Users/kevinhayes/Documents/teaching/images/}}

\usepackage{tikz}
\usetikzlibrary{arrows}

\newcommand{\bbr}{\Bbb{R}}
\newcommand{\zn}{\Bbb{Z}^n}

%\usepackage{epsfig}
%\usepackage{subfigure}
\usepackage{amscd}
\usepackage{amssymb}
\usepackage{amsbsy}
\usepackage{amsthm}
\usepackage{natbib}
\usepackage{amsbsy}
\usepackage{enumerate}
\usepackage{amsmath}
\usepackage{eurosym}
%\usepackage{beamerarticle}
\usepackage{txfonts}
\usepackage{fancyvrb}
\usepackage{fancyhdr}
\usepackage{natbib}
\bibliographystyle{chicago}

\usepackage{vmargin}
% left top textwidth textheight headheight
% headsep footheight footskip
\setmargins{2.0cm}{2.5cm}{16 cm}{22cm}{0.5cm}{0cm}{1cm}{1cm}
\renewcommand{\baselinestretch}{1.3}


\pagenumbering{arabic}
\begin{document}
\section{Cartesian Product / Cross Product}


The Cartesian product of n number of sets $A1,A2,\ldots A_n$ denoted as $A1\times A2 \ldots \times An$ can be defined as all possible ordered pairs $(x1,x2,\ldots x_n)$
where $x1\in A1,x2\in A2,\ldots x_n \in A_n$

\subsection{Example}
If we take two sets A={a,b}A={a,b} and B={1,2}
B={1,2},

\begin{itemize}

\item The Cartesian product of A and B is written 
as  $A \times B={(a,1),(a,2),(b,1),(b,2)}$
\item The Cartesian product of B and A is written as  
$B\times A={(1,a),(1,b),(2,a),(2,b)}$
\end{itemize}
\end{document}
