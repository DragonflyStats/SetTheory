
\documentclass[12pt]{article}
%\usepackage[final]{pdfpages}

\usepackage{graphicx}
\graphicspath{{/Users/kevinhayes/Documents/teaching/images/}}

\usepackage{tikz}
\usetikzlibrary{arrows}

\newcommand{\bbr}{\Bbb{R}}
\newcommand{\zn}{\Bbb{Z}^n}

%\usepackage{epsfig}
%\usepackage{subfigure}
\usepackage{amscd}
\usepackage{amssymb}
\usepackage{amsbsy}
\usepackage{amsthm}
\usepackage{natbib}
\usepackage{amsbsy}
\usepackage{enumerate}
\usepackage{amsmath}
\usepackage{eurosym}
%\usepackage{beamerarticle}
\usepackage{txfonts}
\usepackage{fancyvrb}
\usepackage{fancyhdr}
\usepackage{natbib}
\bibliographystyle{chicago}

\usepackage{vmargin}
% left top textwidth textheight headheight
% headsep footheight footskip
\setmargins{2.0cm}{2.5cm}{16 cm}{22cm}{0.5cm}{0cm}{1cm}{1cm}
\renewcommand{\baselinestretch}{1.3}


\pagenumbering{arabic}
\begin{document}
{Cartesian Product}
{
\begin{itemize}
\item Let $X$ and $Y$ be sets.
\item The \textbf{cartesian product} $X \times Y$ is the set whose elements are \textbf{all} of the ordered pairs of elements $(x,y)$ where $x \in X$ and $y \in Y$.
\end{itemize}

\begin{itemize}
\item Let $X = \{a,b,c\}$
\item Let $Y = \{0,1\}$ 
\item The cartesian product $X \times Y$ is therefore:
\end{itemize}

\begin{itemize}
\item Importantly $X \times Y \neq Y \times X$
\item Recall: Let $X = \{a,b,c\}$ and let $Y = \{0,1\}$ 
\item The cartesian product $Y \times X$ is therefore:
\end{itemize}
}

Discrete Maths

A binary relation on a set A is the collection of ordered pairs of elements of A. In other words, it is the subset of the cartesian product A2= AA

Cartesian Product

This is a direct product pf 2 sets

XY = {(x,y)| xXandyY }

4 suits of cards and 13 Ranks, therefore 52 element cartesian prodcut.

N.B         AB BA
A=A =

Cartesian product is not associative

\section{Cartesian Product}

\begin{itemize}
\item Let $X$ and $Y$ be sets.
\item The \textbf{cartesian product} $X \times Y$ is the set whose elements are \textbf{all} of 
the ordered pairs of elements $(x,y)$ where $x \in X$ and $y \in Y$.
\end{itemize}

\begin{itemize}
\item Let $X = \{a,b,c\}$
\item Let $Y = \{0,1\}$ 
\item The cartesian product $X \times Y$ is therefore:
\end{itemize}

\begin{itemize}
\item Importantly $X \times Y \neq Y \times X$
\item Recall: Let $X = \{a,b,c\}$ and let $Y = \{0,1\}$ 
\item The cartesian product $Y \times X$ is therefore:
\end{itemize}


\section{Cartesian Product / Cross Product}
a Cartesian product is a mathematical operation that returns a set (or product set or simply product) from multiple sets. ... If the Cartesian product rows × columns is taken, the cells of the table contain ordered pairs of the form (row value, column value).

The Cartesian product of n number of sets $A1,A2,\ldots A_n$ denoted as $A1\times A2 \ldots \times An$ can be defined as all possible ordered pairs $(x1,x2,\ldots x_n)$
where $x1\in A1,x2\in A2,\ldots x_n \in A_n$

 Let $ A$ and $ B$ be two sets. The cartesian product of $ A$ and $ B$, denoted  $ A \times B$, is the set of all ordered pairs $ (a,b)$ such that $ a \in A$ and $ b \in B$.
$\displaystyle A \times B = \{ (a,b) \; \vert \; a \in A, \; b \in B \}.$	   

For example, if  $ A=\{ a,b,c \}$ and  $ B= \{ 1,2 \}$, then
$\displaystyle A \times B = \{ (a,1), (a,2), (b,1), (b,2), (c,1), (c,2) \}.$

\subsection{Example}
If we take two sets A={a,b}A={a,b} and B={1,2}
B={1,2},

\begin{itemize}

\item The Cartesian product of A and B is written 
as  $A \times B={(a,1),(a,2),(b,1),(b,2)}$
\item The Cartesian product of B and A is written as  
$B\times A={(1,a),(1,b),(2,a),(2,b)}$
\end{itemize}
\end{document}
