%------------------------------------------------- %

%Slide 2 - How Information is presented - Check

\section{Numbers and Set Theory}

Suppose we have the sets \textit{\textbf{A}} and \textbf{\textit{B}} defined as follows:

\[ \boldsymbol{A} = \{\; \sqrt{2},\; \frac{3}{2},\; 2\; \}\]
\begin{center}
$ \boldsymbol{B} = \{ x \in \mathbb{R} :  x \notin  \mathbb{Q} \}  $
\end{center}

\begin{itemize}
\item[1] $\boldsymbol{A} \cap \mathbb{Q}$
\item[2] $\boldsymbol{A} \cap \boldsymbol{B}$
\item[3] $\boldsymbol{B} \cup \mathbb{Q}$
\end{itemize}



\begin{itemize}
\item $ \mathbb{R}$ Set of all real numbers.
\item $ \mathbb{Q}$ Set of all quotient numbers.
\end{itemize}

\[ \boldsymbol{B} = \{ x \in \mathbb{R} :  x \notin  \mathbb{Q} \}\]  

Set of all real numbers that are not quotients (i.e. numbers that can not be expressed as a division of one integer by another).





\[\boldsymbol{A} \cap \mathbb{Q}\]

