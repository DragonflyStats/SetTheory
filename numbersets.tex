%---------------------------------%
\section{Number Sets}
 The font that the symbols are written in (i.e. $\mathbb{N}$, $\mathbb{R}$) is known as \textit{\textbf{blackboard font}}.
\begin{itemize}
 \item $\mathbb{N}$ Natural Numbers ($0,1,2,3$) 
(Not used in the CIS102 Syllabus)
\item $\mathbb{Z}$ Integers ($-3,-2,-1,0,1,2,3, \ldots$)
\begin{itemize}
\item[$\ast$] $\mathbb{Z}^{+}$ Positive Integers
\item[$\ast$] $\mathbb{Z}^{-}$ Negative Integers
\end{itemize}
\item $\mathbb{Q}$ Rational Numbers
\item $\mathbb{R}$ Real Numbers
\end{itemize}

\section{Numbers and Set Theory}

Suppose we have the sets \textit{\textbf{A}} and \textbf{\textit{B}} defined as follows:

\[ \boldsymbol{A} = \{\; \sqrt{2},\; \frac{3}{2},\; 2\; \}\]
\begin{center}
$ \boldsymbol{B} = \{ x \in \mathbb{R} :  x \notin  \mathbb{Q} \}  $
\end{center}

\begin{itemize}
\item[1] $\boldsymbol{A} \cap \mathbb{Q}$
\item[2] $\boldsymbol{A} \cap \boldsymbol{B}$
\item[3] $\boldsymbol{B} \cup \mathbb{Q}$
\end{itemize}



\begin{itemize}
\item $ \mathbb{R}$ Set of all real numbers.
\item $ \mathbb{Q}$ Set of all quotient numbers.
\end{itemize}

\[ \boldsymbol{B} = \{ x \in \mathbb{R} :  x \notin  \mathbb{Q} \}\]  

Set of all real numbers that are not quotients (i.e. numbers that can not be expressed as a division of one integer by another).





\[\boldsymbol{A} \cap \mathbb{Q}\]


% \newpage
\section*{Part D: Natural, Rational and Real Numbers}
\begin{framed}
\begin{itemize}
\item $\mathbb{N}$ : natural numbers (or positive integers) $\{1,2,3,\ldots\}$
\item $\mathbb{Z}$ : integers $\{-3,-2,-1,0,1,2,3,\ldots\}$
\begin{itemize}
\item[$\ast$] (The letter $\mathbb{Z}$ comes from the word \emph{Zahlen} which means ``numbers" in German.)
\end{itemize}
\item $\mathbb{Q}$ : rational numbers
\item $\mathbb{R}$ : real numbers
\item $\mathbb{N} \subseteq \mathbb{Z } \subseteq \mathbb{Q} \subseteq \mathbb{R}$
\begin{itemize}
\item[$\ast$] (All natural numbers are integers. All integers are rational numbers. All rational numbers are real numbers.)
\end{itemize}
\end{itemize}
\end{framed}

\begin{enumerate}
\item State which of the following sets the following numbers belong to. 
  \begin{multicols}{4}
    \begin{itemize}
    \item[1)] $18$
    \item[2)] $8.2347\ldots$
    \item[3)] $\pi$
    \item[4)] $1.33333\ldots$
    \item[5)] $17/4$
    \item[6)] $4.25$
    \item[7)] $\sqrt{\pi}$
    \item[8)] $\sqrt{25}$
    %\item[i)]
    %\item[j)] $(15)_{10}$
    %\item[k)] $\pi$
    %\item[l)] $11.132$
    \end{itemize}
  \end{multicols}
  \bigskip
  The possible answers are

    \begin{itemize}
    \item[a)] Natural number : $\mathbb{N} \subseteq \mathbb{Z } \subseteq \mathbb{Q} \subseteq \mathbb{R}$
    \item[b)] Integer : $ \mathbb{Z } \subseteq \mathbb{Q} \subseteq \mathbb{R}$
    \item[c)] Rational Number : $ \mathbb{Q} \subseteq \mathbb{R}$
    \item[d)] Real Number $\mathbb{R}$

    %\item[i)]
    %\item[j)] $(15)_{10}$
    %\item[k)] $\pi$
    %\item[l)] $11.132$
    \end{itemize}

\end{enumerate}

