%---------------------------------%
\section{Number Sets}
 The font that the symbols are written in (i.e. $\mathbb{N}$, $\mathbb{R}$) is known as \textit{\textbf{blackboard font}}.
\begin{itemize}
 \item $\mathbb{N}$ Natural Numbers ($0,1,2,3$) 
(Not used in the CIS102 Syllabus)
\item $\mathbb{Z}$ Integers ($-3,-2,-1,0,1,2,3, \ldots$)
\begin{itemize}
\item[$\ast$] $\mathbb{Z}^{+}$ Positive Integers
\item[$\ast$] $\mathbb{Z}^{-}$ Negative Integers
\end{itemize}
\item $\mathbb{Q}$ Rational Numbers
\item $\mathbb{R}$ Real Numbers
\end{itemize}

\section{Numbers and Set Theory}

Suppose we have the sets \textit{\textbf{A}} and \textbf{\textit{B}} defined as follows:

\[ \boldsymbol{A} = \{\; \sqrt{2},\; \frac{3}{2},\; 2\; \}\]
\begin{center}
$ \boldsymbol{B} = \{ x \in \mathbb{R} :  x \notin  \mathbb{Q} \}  $
\end{center}

\begin{itemize}
\item[1] $\boldsymbol{A} \cap \mathbb{Q}$
\item[2] $\boldsymbol{A} \cap \boldsymbol{B}$
\item[3] $\boldsymbol{B} \cup \mathbb{Q}$
\end{itemize}



\begin{itemize}
\item $ \mathbb{R}$ Set of all real numbers.
\item $ \mathbb{Q}$ Set of all quotient numbers.
\end{itemize}

\[ \boldsymbol{B} = \{ x \in \mathbb{R} :  x \notin  \mathbb{Q} \}\]  

Set of all real numbers that are not quotients (i.e. numbers that can not be expressed as a division of one integer by another).





\[\boldsymbol{A} \cap \mathbb{Q}\]


% \newpage
\section*{Part D: Natural, Rational and Real Numbers}
\begin{framed}
\begin{itemize}
\item $\mathbb{N}$ : natural numbers (or positive integers) $\{1,2,3,\ldots\}$
\item $\mathbb{Z}$ : integers $\{-3,-2,-1,0,1,2,3,\ldots\}$
\begin{itemize}
\item[$\ast$] (The letter $\mathbb{Z}$ comes from the word \emph{Zahlen} which means ``numbers" in German.)
\end{itemize}
\item $\mathbb{Q}$ : rational numbers
\item $\mathbb{R}$ : real numbers
\item $\mathbb{N} \subseteq \mathbb{Z } \subseteq \mathbb{Q} \subseteq \mathbb{R}$
\begin{itemize}
\item[$\ast$] (All natural numbers are integers. All integers are rational numbers. All rational numbers are real numbers.)
\end{itemize}
\end{itemize}
\end{framed}

\begin{enumerate}
\item State which of the following sets the following numbers belong to. 
  \begin{multicols}{4}
    \begin{itemize}
    \item[1)] $18$
    \item[2)] $8.2347\ldots$
    \item[3)] $\pi$
    \item[4)] $1.33333\ldots$
    \item[5)] $17/4$
    \item[6)] $4.25$
    \item[7)] $\sqrt{\pi}$
    \item[8)] $\sqrt{25}$
    %\item[i)]
    %\item[j)] $(15)_{10}$
    %\item[k)] $\pi$
    %\item[l)] $11.132$
    \end{itemize}
  \end{multicols}
  \bigskip
  The possible answers are

    \begin{itemize}
    \item[a)] Natural number : $\mathbb{N} \subseteq \mathbb{Z } \subseteq \mathbb{Q} \subseteq \mathbb{R}$
    \item[b)] Integer : $ \mathbb{Z } \subseteq \mathbb{Q} \subseteq \mathbb{R}$
    \item[c)] Rational Number : $ \mathbb{Q} \subseteq \mathbb{R}$
    \item[d)] Real Number $\mathbb{R}$

    %\item[i)]
    %\item[j)] $(15)_{10}$
    %\item[k)] $\pi$
    %\item[l)] $11.132$
    \end{itemize}

\end{enumerate}



%---------------------------%
\end{itemize}
\section*{Part E: Natural, Rational and Real Numbers}
\begin{framed}
\end{framed}

\begin{enumerate}
\item State which of the following sets the following numbers belong to. 
\begin{multicols}{4}
\begin{itemize}
\item[1)] $18$
\item[2)] $8.2347\ldots$
\item[3)] $\pi$
\item[4)] $1.33333\ldots$
\item[5)] $17/4$
\item[6)] $4.25$
\item[7)] $\sqrt{\pi}$
\item[8)] $\sqrt{25}$
%\item[i)]
%\item[j)] $(15)_{10}$
%\item[k)] $\pi$
%\item[l)] $11.132$
\end{itemize}
\end{multicols}
\bigskip
The possible answers are

\begin{itemize}
\item[a)] Natural number : $\mathbb{N} \subseteq \mathbb{Z } \subseteq \mathbb{Q} \subseteq \mathbb{R}$
\item[b)] Integer : $ \mathbb{Z } \subseteq \mathbb{Q} \subseteq \mathbb{R}$
\item[c)] Rational Number : $ \mathbb{Q} \subseteq \mathbb{R}$
\item[d)] Real Number $\mathbb{R}$
%\item[i)]
%\item[j)] $(15)_{10}$
%\item[k)] $\pi$
%\item[l)] $11.132$
\end{itemize}

\end{enumerate}

\begin{itemize}
\item $\mathbb{N}$ : natural numbers (or positive integers) $\{1,2,3,\ldots\}$
\item $\mathbb{Z}$ : integers $\{-3,-2,-1,0,1,2,3,\ldots\}$
\begin{itemize}
\item[$\ast$] (The letter $\mathbb{Z}$ comes from the word \emph{Zahlen} which means ``numbers" in German.)
\end{itemize}
\item $\mathbb{Q}$ : rational numbers
\item $\mathbb{R}$ : real numbers
\item $\mathbb{N} \subseteq \mathbb{Z } \subseteq \mathbb{Q} \subseteq \mathbb{R}$
\begin{itemize}
\item[$\ast$] (All natural numbers are integers. All integers are rational numbers. All rational numbers are real numbers.)
\end{itemize}
\end{itemize}




% \newpage
\section*{Part D: Natural, Rational and Real Numbers}
\begin{framed}
	\begin{itemize}
		\item $\mb{N}$ : natural numbers (or positive integers) $\{1,2,3,\ldots\}$
		\item $\mb{Z}$ : integers $\{-3,-2,-1,0,1,2,3,\ldots\}$
		\begin{itemize}
			\item[$\ast$] (The letter $\mb{Z}$ comes from the word \emph{Zahlen} which means ``numbers" in German.)
		\end{itemize}
		\item $\mb{Q}$ : rational numbers
		\item $\mb{R}$ : real numbers
		\item $\mb{N} \subseteq \mb{Z } \subseteq \mb{Q} \subseteq \mb{R}$
		\begin{itemize}
			\item[$\ast$] (All natural numbers are integers. All integers are rational numbers. All rational numbers are real numbers.)
		\end{itemize}
	\end{itemize}
\end{framed}

\begin{enumerate}
	\item State which of the following sets the following numbers belong to. 
	\begin{multicols}{4}
		\begin{itemize}
			\item[1)] $18$
			\item[2)] $8.2347\ldots$
			\item[3)] $\pi$
			\item[4)] $1.33333\ldots$
			\item[5)] $17/4$
			\item[6)] $4.25$
			\item[7)] $\sqrt{\pi}$
			\item[8)] $\sqrt{25}$
			%\item[i)]
			%\item[j)] $(15)_{10}$
			%\item[k)] $\pi$
			%\item[l)] $11.132$
		\end{itemize}
	\end{multicols}
	\bigskip
	The possible answers are
	
	\begin{itemize}
		\item[a)] Natural number : $\mb{N} \subseteq \mb{Z } \subseteq \mb{Q} \subseteq \mb{R}$
		\item[b)] Integer : $ \mb{Z } \subseteq \mb{Q} \subseteq \mb{R}$
		\item[c)] Rational Number : $ \mb{Q} \subseteq \mb{R}$
		\item[d)] Real Number $\mb{R}$
		
		%\item[i)]
		%\item[j)] $(15)_{10}$
		%\item[k)] $\pi$
		%\item[l)] $11.132$
	\end{itemize}
	
\end{enumerate}
%---------------------------------%
\section*{Number Sets}
The font that the following symbols are written in (i.e. $\mathbb{N}$, $\mathbb{R}$) is known as \textit{\textbf{blackboard font}}.
\begin{itemize}
	\item $\mathbb{N}$ Natural Numbers ($1,2,3,\ldots$) 
%	\textit{(Not used in the CIS102 Syllabus)}
	\item $\mathbb{Z}$ Integers ($-3,-2,-1,0,1,2,3, \ldots$)
	\begin{itemize}
		\item[$\bullet$] $\mathbb{Z}^{+}$ Positive Integers
		\item[$\bullet$] $\mathbb{Z}^{-}$ Negative Integers
		\item[$\bullet$] 0 is not considered as either positive or negative.
	\end{itemize}
	\item $\mathbb{Q}$ Rational Numbers
	\item $\mathbb{R}$ Real Numbers
	\item $\mathbb{C}$ Complex Numbers
\end{itemize}


\section*{Part D: Natural, Rational and Real Numbers}
\begin{itemize}
	\item $\mathbb{N}$ : natural numbers (or positive integers) $\{1,2,3,\ldots\}$
	\item $\mathbb{Z}$ : integers $\{-3,-2,-1,0,1,2,3,\ldots\}$
	\begin{itemize}
		\item (The letter $\mathbb{Z}$ comes from the word \emph{Zahlen} which means ``numbers" in German.)
	\end{itemize}
	\item $\mathbb{Q}$ : rational numbers
	\item $\mathbb{R}$ : real numbers
	\item $\mathbb{N} \subseteq \mathbb{Z } \subseteq \mathbb{Q} \subseteq \mathbb{R}$
	\begin{itemize}
		\item (All natural numbers are integers. All integers are rational numbers. All rational numbers are real numbers.)
	\end{itemize}
\end{itemize}
\newpage


\section{Special sets of numbers}
\smallskip 
It is convenient to denote certain key sets of numbers by a standard letter.
Definition 2.2 
The symbol Z is used to denote the set of integers," 
 denotes the set of positive integers; 
 R denotes the set of real numbers;
Q denotes the set of rational numbers {the letter Q stands for “quotient”).
\smallskip 
%-------------------------------------- %
\smallskip 
Of these sets, we can specify the set of positive integers and the set of integers by the listing method
using elipses, as follows:
\[Z^{+} = {1,2,s,...}\]
\[2 = {o,1,-1,2,-2,3,-2,...}\]
Note that Z includes 0 and the negative whole numbers as well as the positive ones. Similarly,
\mathbb{R} contains O and the negative reals and Q contains 0 and the negative rationals, as well as the
positive ones.

%----------------------------------------- %
%=====================================================%
\subsection{Some Important Sets}
\begin{itemize}
\item N − the set of all natural numbers = {1,2,3,4,.....}
\item Z − the set of all integers = {.....,−3,−2,−1,0,1,2,3,.....}
\item Z+ − the set of all positive integers
\item Q − the set of all rational numbers
\item R − the set of all real numbers
\item W − the set of all whole numbers
\end{itemize}


%---------------------------------%
\subsection{Number Sets}
The font that the symbols are written in (i.e. $\mathbb{N}$, $\mathbb{R}$) is known as \textit{\textbf{blackboard font}}.
\begin{itemize}
\item $\mathbb{N}$ Natural Numbers ($0,1,2,3$) 
(Not used in the CIS102 Syllabus)
\item $\mathbb{Z}$ Integers ($-3,-2,-1,0,1,2,3, \ldots$)
\begin{itemize}
\item[$\ast$] $\mathbb{Z}^{+}$ Positive Integers
\item[$\ast$] $\mathbb{Z}^{-}$ Negative Integers
\end{itemize}
\item $\mathbb{Q}$ Rational Numbers
\item $\mathbb{R}$ Real Numbers
\end{itemize}

%---------------------------%
\newpage

%---------------------------------------------------%
%---------------------------------------------------%
% Schaum 1:4


{Set Operations}
Let $U = \{1,2,\ldots, 9\}$ be the universal set, and let
\begin{itemize}
\item A = $\{1, 2, 3, 4, 5\}$,  
\item B = $\{4, 5, 6, 7\}$,  
\item C = $\{5, 6, 7, 8, 9\}$
\item D = $\{1, 3, 5, 7, 9\}$,
\item E = $\{2, 4, 6, 8\}$,
\item F = $\{1, 5, 9\}$.
\end{itemize}

%---------------------------------------------------%
%---------------------------------------------------%
% Schaum 1:4

{Set Operations}

%%- \vspace{-3cm}
Find: 
\begin{itemize}
\item[(a)] A $\cup$ B and A $\cap$ B, 
\item[(b)] C $\cup$ D and C $\cap$ D, 
\item[(c)] E $\cup$ F and E $\cap$ F.
\end{itemize}


%---------------------------------------------------%
%---------------------------------------------------%
% Schaum 1:4

{Set Operations}

%%- \vspace{-3cm}
Find: 
\begin{itemize}
\item[(a)] A $\cup$ B and A $\cap$ B, \bigskip
\item A = $\{1, 2, 3, 4, 5\}$  
\item B = $\{4, 5, 6, 7\}$  
\end{itemize}


%---------------------------------------------------%
%---------------------------------------------------%
% Schaum 1:4

{Set Operations}

%%- \vspace{-3cm}
Find: 
\begin{itemize}
\item[(b)] C $\cup$ D and C $\cap$ D,  \bigskip
\item C = $\{5, 6, 7, 8, 9\}$
\item D = $\{1, 3, 5, 7, 9\}$
\end{itemize}


%---------------------------------------------------%
%---------------------------------------------------%
% Schaum 1:4

{Set Operations}
%%- \vspace{-3cm}

Find: 
\begin{itemize}
\item[(c)] E $\cup$ F and E $\cap$ F. \bigskip
\item E = $\{2, 4, 6, 8\}$

\item F = $\{1, 5, 9\}$
\end{itemize}


%---------------------------------------------------%
%---------------------------------------------------%
%SLIDE SET 3 : FINITE SETS
%---------------------------------------------------%
%---------------------------------------------------%
% Opening Slide 3


\Huge
\[\mbox{Discrete Mathematics}\]
\Huge
\[\mbox{Set Theory : Finite Sets}\]


\[\mbox{www.Stats-Lab.com}\]

\[\mbox{Twitter: @StatsLabDublin}\]


%---------------------------------------------------%

{Finite Sets}

%%- \vspace{-1.2cm}
Determine which of the following sets are finite:
\begin{itemize}
\item[(a)] Set of Prime numbers %infinite.
%%- \vspace{0.4cm}
\item[(b)] Set of two digit Prime numbers %infinite.
%\item[(a)] Lines parallel to the x axis. 
%%- \vspace{0.4cm}
\item[(c)] Letters in the English alphabet.
%%- \vspace{0.4cm}
\item[(d)] Integers which are multiples of 5.
%%- \vspace{0.4cm}
%\item[(c)] Letters of the English alphabet
\item[(e)] Days of the week % finite sets.
%\item[(e)] The number of grains of rice in a ton of the grain
\end{itemize}





%---------------------------------------------------%
%---------------------------------------------------%
%SLIDE SET 4
%---------------------------------------------------%
%---------------------------------------------------%
% Opening Slide 4


\Huge
\[\mbox{Discrete Mathematics}\]
\Huge
\[\mbox{Set Theory}\]


\[\mbox{www.Stats-Lab.com}\]

\[\mbox{Twitter: @StatsLabDublin}\]



%---------------------------------------------------%

{Set Theory}

%%- \vspace{-1.8cm}
Given the set \textbf{A} is contructed as follows 
\[ [\{a, b\}, \{c\}, \{d, e, f \} ]. \]

\begin{itemize}
\item[(a)] List the elements of \textbf{A}. 
\item[(b)] Find the cardinality of \textbf{A} : $n(\boldsymbol{A})$. 
\item[(c)] Find the power set of \textbf{A}.
\end{itemize}



%---------------------------------------------------%

{Set Theory}

%%- \vspace{-3.8cm}
\[\boldsymbol{A} = [\{a, b\}, \{c\},\{d, e, f \}]. \]
\begin{itemize}
\item[(a)] List the elements of \textbf{A}. 
\end{itemize}



%---------------------------------------------------%
%---------------------------------------------------%

{Set Theory}

%%- \vspace{-3.8cm}
\[\boldsymbol{A} = [\{a, b\}, \{c\},\{d, e, f \}]. \]
\begin{itemize}
\item[(b)] Find the cardinality of \textbf{A} : $n(\boldsymbol{A})$. 
\end{itemize}



%---------------------------------------------------%
%---------------------------------------------------%

{Set Theory}

%%- \vspace{-3.8cm}
\[\boldsymbol{A} = [\{a, b\}, \{c\},\{d, e, f \}]. \]
\begin{itemize}
\item[(c)] Find the power set of \textbf{A}. 
\end{itemize}



%---------------------------------------------------%
%---------------------------------------------------%
%SLIDE SET 5
%---------------------------------------------------%
%---------------------------------------------------%
% Opening Slide 5


\Huge
\[\mbox{Discrete Mathematics}\]
\Huge
\[\mbox{Set Theory}\]


\[\mbox{www.Stats-Lab.com}\]

\[\mbox{Twitter: @StatsLabDublin}\]




Consider the set \textbf{A}, which is a subset of the the universal set of real numbers $\mathbb{R}$
 \[\boldsymbol{A} = \{4, \sqrt{2}, 2/3, -2.5, -5, 33, \sqrt{9}, \pi \}\]
Using formal set notation, write the sets of:
\begin{itemize}
\item[(a)] natural numbers in \textbf{A}
\item[(b)] integers in \textbf{A}
\item[(c)] rational numbers in \textbf{A}
\item[(d)] irrational numbers in \textbf{A}
\end{itemize}

%---------------------------------------------------------- %

%0 Reveals

%%- \vspace{-0.5cm}
\[\boldsymbol{A} = \{4,\; \sqrt{2},\; 2/3,\; -2.5,\; -5,\; 33,\; \sqrt{9},\; \pi \}\]
%%- \vspace{-0.5cm}
\begin{itemize}
\item[(a)] natural numbers in \textbf{A}\\
\phantom{Answer: $\{4,\; 33,\; \sqrt{9}\}$}
\item[(b)] integers in \textbf{A}\\
\phantom{Answer: $\{4,\; -5, 33,\; \sqrt{9}\}$}
\item[(c)] rational numbers in \textbf{A}\\
\phantom{Answer: $\{4,\; 2/3,\; -2.5,\; -5,\; 33,\; \sqrt{9}\}$}
\item[(d)] irrational numbers in \textbf{A}\\
\phantom{Answer: $\{\sqrt{2},\; \pi\}$}
\end{itemize}


%---------------------------------------------------------- %

%1 Reveals

%%- \vspace{-0.5cm}
\[\boldsymbol{A} = \{4,\; \sqrt{2},\; 2/3,\; -2.5,\; -5,\; 33,\; \sqrt{9},\; \pi \}\]
%%- \vspace{-0.5cm}
\begin{itemize}
\item[(a)] natural numbers in \textbf{A}\\
\hspace{1cm} \textit{Answer}: $\{4,\; 33,\; \sqrt{9}\}$
\item[(b)] integers in \textbf{A}\\
\phantom{Answer: $\{4,\; -5, 33,\; \sqrt{9}\}$}
\item[(c)] rational numbers in \textbf{A}\\
\phantom{Answer: $\{4,\; 2/3,\; -2.5,\; -5,\; 33,\; \sqrt{9}\}$}
\item[(d)] irrational numbers in \textbf{A}\\
\phantom{Answer: $\{\sqrt{2},\; \pi\}$}
\end{itemize}


%---------------------------------------------------------- %

%2 Reveals

%%- \vspace{-0.5cm}
\[\boldsymbol{A} = \{4,\; \sqrt{2},\; 2/3,\; -2.5,\; -5,\; 33,\; \sqrt{9},\; \pi \}\]
%%- \vspace{-0.5cm}
\begin{itemize}
\item[(a)] natural numbers in \textbf{A}\\
\hspace{1cm} \textit{Answer}: $\{4,\; 33,\; \sqrt{9}\}$
\item[(b)] integers in \textbf{A}\\
\hspace{1cm} \textit{Answer}: $\{4,\; -5,\; 33,\; \sqrt{9}\}$
\item[(c)] rational numbers in \textbf{A}\\
\phantom{Answer: $\{4,\; 2/3,\; -2.5,\; -5,\; 33,\; \sqrt{9}\}$}
\item[(d)] irrational numbers in \textbf{A}\\
\phantom{Answer: $\{\sqrt{2},\; \pi\}$}
\end{itemize}


%---------------------------------------------------------- %

%3 Reveals

%%- \vspace{-0.5cm}
\[\boldsymbol{A} = \{4,\; \sqrt{2},\; 2/3,\; -2.5,\; -5,\; 33,\; \sqrt{9},\; \pi \}\]
%%- \vspace{-0.5cm}
\begin{itemize}
\item[(a)] natural numbers in \textbf{A}\\
\hspace{1cm} \textit{Answer}: $\{4,\; 33,\; \sqrt{9}\}$
\item[(b)] integers in \textbf{A}\\
\hspace{1cm} \textit{Answer}: $\{4,\; -5,\; 33,\; \sqrt{9}\}$
\item[(c)] rational numbers in \textbf{A}\\
\hspace{1cm} \textit{Answer}: $\{4,\; 2/3,\; -2.5,\; -5,\; 33,\; \sqrt{9}\}$
\item[(d)] irrational numbers in \textbf{A}\\
\phantom{Answer: $\{\sqrt{2},\; \pi\}$}
\end{itemize}

%---------------------------------------------------------- %

%4 Reveals

%%- \vspace{-0.5cm}
\[\boldsymbol{A} = \{4,\; \sqrt{2},\; 2/3,\; -2.5,\; -5,\; 33,\; \sqrt{9},\; \pi \}\]
%%- \vspace{-0.5cm}
\begin{itemize}
\item[(a)] natural numbers in \textbf{A}\\
\hspace{1cm} \textit{Answer}: $\{4,\; 33,\; \sqrt{9}\}$
\item[(b)] integers in \textbf{A}\\
\hspace{1cm} \textit{Answer}: $\{4,\; -5,\; 33,\; \sqrt{9}\}$
\item[(c)] rational numbers in \textbf{A}\\
\hspace{1cm} \textit{Answer}: $\{4,\; 2/3,\; -2.5,\; -5,\; 33,\; \sqrt{9}\}$
\item[(d)] irrational numbers in \textbf{A}\\
\hspace{1cm} \textit{Answer}: $\{\sqrt{2},\; \pi\}$
\end{itemize}



%----------------------------------------------------------------%

\subsection{Number Sets}

%% %% - \vspace{-1cm}
\textbf{Blackboard Bold Typeface}

\begin{itemize}
\item Conventionally the symbols for numbers sets are written in a special typeface, known as \textbf{blackboard bold}.
\item Examples : $\mathbb{N}$, $\mathbb{Z}$ and $\mathbb{R}$.

\end{itemize}


%% %% - \vspace{-1cm}
\textbf{Natural Numbers} ($\mathbb{N}$)
\begin{itemize}
\item The whole numbers from 1 upwards. 

\item The set of natural numbers is 
\[\{1,2,3,4,5,6,\ldots\} \]
\item In some branches of mathematics, $0$ might be counted as a natural number.
\[\{0,1,2,3,4,5,6,\ldots\} \]
\end{itemize}

%----------------------------------------------------------------%


\subsection{Number Sets}


%% %% - \vspace{-0.5cm}
\textbf{Integers} ($\mathbb{Z}$)
\begin{itemize}
\item The integers are all the whole numbers, all the negative whole numbers and zero.

\item The set of integers is 
\[\{\ldots,-4,-3,-2,-1,\;0,\;1,\;2,\;3,\ldots\} \]
\item The notation $\mathbb{Z}$ is from the German word for numbers: \textit{Zahlen}. 
\item All natural numbers are integers.
\[ \mathbb{Q}  \subset \mathbb{Z}\]
\end{itemize}

%----------------------------------------------------------------%


\subsection{Number Sets}

%% %% - \vspace{-1.8cm}
\textbf{Integers} ($\mathbb{Z}$)
\begin{itemize}
\item Natural numbers may also be referred to as positive integers, denoted $\mathbb{Z}^{+}$. \\(note the superscript)
\item Negative integers are denoted $\mathbb{Z}^{-}$.
\[\{\ldots,-4,-3,-2,-1\}\]
\end{itemize}

%----------------------------------------------------------------%


\subsection{Number Sets}

%% %% - \vspace{-1.8cm}
\textbf{Integers} ($\mathbb{Z}$)
\begin{itemize}
\item 0 is neither positive nor negative. The following set of non-negative numbers \[\{0,1,2,3,4,5,6,\ldots\} \] might be denoted $0 \cup \mathbb{Z}^{+}$
\item $\cup$ is the mathematical symbol for \textbf{union}.
\end{itemize}



\subsection{Number Sets}

%% %% - \vspace{-1cm}
\textbf{Rational Numbers} ($\mathbb{Q}$)
\begin{itemize}
\item Rational numbers, also known as quotients, are numbers you can make by dividing one integer by another (but not dividing by zero). 
\item If a number can be expressed as one integer divided by another, it is a rational number.
\[ \mathbb{Q} = \left\{\; \frac{p}{q} \;\bigg| p \in \mathbb{Z},\; q \in \mathbb{Z},\; q \neq 0  \;   \right\}   \]
\end{itemize}

%-------------------------------------------------------------- %   

\subsection{Number Sets}

%% %% - \vspace{-1cm}
\textbf{Rational Numbers} ($\mathbb{Q}$)
\begin{itemize}
\item All integers are rational numbers 
\[ \mathbb{Z}  \subset \mathbb{Q}\]
(and by extension all natural numbers are rational numbers too)
\item Examples of rational numbers
\[ 9500,\;7,\; \frac{1}{2} ,\; \frac{3}{7},\; -2.6 ,\; 0.001\] 
\end{itemize}

%-------------------------------------------------------------- %   

\subsection{Number Sets}

%% %% - \vspace{-1cm}
\textbf{Irrational Numbers} 
\begin{itemize}
\item A number that can not be written as the ratio of two integers is known as an irrational number.
\item Two famous examples of irrational numbers are $\pi$ and $\sqrt{2}$. 
\[\pi = 3.141592\ldots\]
\[\sqrt{2} = 1.41421\ldots\]
\end{itemize}


%% %% - \vspace{-1cm}
\textbf{Real Numbers} ($\mathbb{R}$)
\begin{itemize}
\item Irrational numbers are types of real numbers.
\item Rational numbers are real numbers too.
\[ \mathbb{Q}  \subset \mathbb{R}\]

\item A real number is simply any point anywhere on the number line.
\end{itemize}


%% %% - \vspace{-1cm}
\textbf{Real Numbers} ($\mathbb{R}$)
\begin{itemize}
\item There are numbers that are not real numbers, for example \textbf{imaginary numbers}, but we will not cover them in this presentation.
\end{itemize}


\end{document}
