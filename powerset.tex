
\documentclass[11pt,a4paper,titlepage,oneside,openany]{article}

\pagestyle{plain}
%\renewcommand{\baselinestretch}{1.7}

\usepackage{setspace}
%\singlespacing
\onehalfspacing
%\doublespacing
%\setstretch{1.1}

\usepackage{amsmath}
\usepackage{amssymb}
\usepackage{amsthm}
\usepackage{framed}
\usepackage{multicol}

\usepackage[margin=3cm]{geometry}
\usepackage{graphicx,psfrag}%\usepackage{hyperref}
\usepackage[small]{caption}
\usepackage{subfig}

\usepackage{algorithm}
\usepackage{algorithmic}
\newcommand{\theHalgorithm}{\arabic{algorithm}}

\usepackage{varioref} %NB: FIGURE LABELS MUST ALWAYS COME DIRECTLY AFTER CAPTION!!!
%\newcommand{\vref}{\ref}

\usepackage{index}
\makeindex
\newindex{sym}{adx}{and}{Symbol Index}
%\newcommand{\symindex}{\index[sym]}
%\newcommand{\symindex}[1]{\index[sym]{#1}\hfill}
\newcommand{\symindex}[1]{\index[sym]{#1}}

%\usepackage[breaklinks,dvips]{hyperref}%Always put after varioref, or you'll get nested section headings
%Make sure this is after index package too!
%\hypersetup{colorlinks=false,breaklinks=true}
%\hypersetup{colorlinks=false,breaklinks=true,pdfborder={0 0 0.15}}


%\usepackage{breakurl}

\graphicspath{{./images/}}

\usepackage[subfigure]{tocloft}%For table of contents
\setlength{\cftfignumwidth}{3em}

\input{longdiv}
\usepackage{wrapfig}


%\usepackage{index}
%\makeindex
%\usepackage{makeidx}

%\usepackage{lscape}
\usepackage{pdflscape}
\usepackage{multicol}

\usepackage[utf8]{inputenc}

%\usepackage{fullpage}

%Compulsory packages for the PhD in UL:
%\usepackage{UL Thesis}
\usepackage{natbib}

%\numberwithin{equation}{section}
\numberwithin{equation}{section}
\numberwithin{algorithm}{section}
\numberwithin{figure}{section}
\numberwithin{table}{section}
%\newcommand{\vec}[1]{\ensuremath{\math{#1}}}

%\linespread{1.6} %for double line spacing

\usepackage{afterpage}%fingers crossed

\newtheorem{thm}{Theorem}[section]
\newtheorem{defin}{Definition}[section]
\newtheorem{cor}[thm]{Corollary}
\newtheorem{lem}[thm]{Lemma}

%\newcommand{\dbar}{{\mkern+3mu\mathchar'26\mkern-12mu d}}
\newcommand{\dbar}{{\mkern+3mu\mathchar'26\mkern-12mud}}

\newcommand{\bbSigma}{{\mkern+8mu\mathsf{\Sigma}\mkern-9mu{\Sigma}}}
\newcommand{\thrfor}{{\Rightarrow}}

\newcommand{\mb}{\mathbb}
\newcommand{\bx}{\vec{x}}
\newcommand{\bxi}{\boldsymbol{\xi}}
\newcommand{\bdeta}{\boldsymbol{\eta}}
\newcommand{\bldeta}{\boldsymbol{\eta}}
\newcommand{\bgamma}{\boldsymbol{\gamma}}
\newcommand{\bTheta}{\boldsymbol{\Theta}}
\newcommand{\balpha}{\boldsymbol{\alpha}}
\newcommand{\bmu}{\boldsymbol{\mu}}
\newcommand{\bnu}{\boldsymbol{\nu}}
\newcommand{\bsigma}{\boldsymbol{\sigma}}
\newcommand{\bdiff}{\boldsymbol{\partial}}

\newcommand{\tomega}{\widetilde{\omega}}
\newcommand{\tbdeta}{\widetilde{\bdeta}}
\newcommand{\tbxi}{\widetilde{\bxi}}



\newcommand{\wv}{\vec{w}}

\newcommand{\ie}{i.e. }
\newcommand{\eg}{e.g. }
\newcommand{\etc}{etc}

\newcommand{\viceversa}{vice versa}
\newcommand{\FT}{\mathcal{F}}
\newcommand{\IFT}{\mathcal{F}^{-1}}
%\renewcommand{\vec}[1]{\boldsymbol{#1}}
\renewcommand{\vec}[1]{\mathbf{#1}}
\newcommand{\anged}[1]{\langle #1 \rangle}
\newcommand{\grv}[1]{\grave{#1}}
\newcommand{\asinh}{\sinh^{-1}}

\newcommand{\sgn}{\text{sgn}}
\newcommand{\morm}[1]{|\det #1 |}

\newcommand{\galpha}{\grv{\alpha}}
\newcommand{\gbeta}{\grv{\beta}}
%\newcommand{\rnlessO}{\mb{R}^n \setminus \vec{0}}
\usepackage{listings}

\interfootnotelinepenalty=10000

\newcommand{\sectionline}{%
  \nointerlineskip \vspace{\baselineskip}%
  \hspace{\fill}\rule{0.5\linewidth}{.7pt}\hspace{\fill}%
  \par\nointerlineskip \vspace{\baselineskip}
}

\renewcommand{\labelenumii}{\roman{enumii})}

\begin{document}

\textbf{Power set}

The power set of X, $P(X)$, is the set whose elements are all the subsets of X. Thus \[P(A) = \{ \{\}, \{1\}, \{2\}, \{3\}, \{1,2\}, \{1,3\}, \{2,3\}, \{1,2,3\}\}\]. The power set of the empty set $P(\{\})$ = $\{\{\}\}$. 

Note that in both cases the cardinality of the power set is strictly greater than that of base set: No one-to-one correspondence exists between the set and its power set. 

%Cantor proved that this in fact holds for any set (Cantor's Theorem). This is obvious for a finite set, but Cantor's ingenious proof made no reference to the set being finite; the theorem holds even for infinite sets. This was a powerful generalisation of his previous discovery that different sizes of infinity exist.

\newpage



\subsubsection*{Power Sets}

\begin{itemize}

\item Consider the set A where $ A = \{w,x,y,z\}$
\item There are 4 elements in set A.

\item The power set of A contains 16 element data sets.

\item \[  \mathcal{P}(A) = \{\{ x \}, \{ y \} \}  \]

\item (i.e. 1 null set, 4 single element sets, 6 two -elemnts sets, 4 three lement set and one 4- element set.)

\end{itemize}

%-------------------------------------------------% 

\section*{Power Sets}
\subsection*{Worked Example}
Consider the set $Z$:
\[ Z = \{ a,b,c\}  \]
\begin{itemize}
\item[Q1] How many sets are in the power set of $Z$? 
\item[Q2] Write out the power set of $Z$. 
\item[Q3] How many elements are in each element set?
\end{itemize}
%----------------------------------------------%
\subsection*{Solutions to Worked Example}

\begin{itemize}


\item[Q1] There are 3 elements in $Z$. So there is $2^3 = 8$ element sets contained in the power set.

\item[Q2] Write out the power set of $Z$.
\[ \mathcal{P}(Z) = \{ \{0\}, \{a\}, \{b\}, \{c\}, \{a,b\}, \{a,c\}, \{b,c\}, \{a,b,c\} \]

\item[Q3]
\begin{itemize}
\item[*] One element set is the null set - i.e. containing no
elements \item[$\bullet$] Three element sets have only elements \item[$\bullet$]
Three element sets have two elements \item[$\bullet$] One element set
contains all three elements \item[$\bullet$] 1+3+3+1=8
\end{itemize}
\end{itemize}
\subsection*{Exercise}
For the set $Y = \{u,v,w,x\}$ , answer the questions from the
previous exercise
\end{document}


 
Power Sets
Worked Example
Consider the set Z:  Z = { a,b,c}  
Q1 How many sets are in the power set of Z?
Q2 Write out the power set of Z.
Q3 How many elements are in each element set?

Solutions to Worked Example
Q1 answer:  There are 3 elements in Z. So there is 23 = 8 element sets contained in the power set.
Q2 answer:  Write out the power set of Z.
P(Z) = { {0}, {a}, {b}, {c}, {a,b}, {a,c}, {b,c}, {a,b,c} }

Q3 answer:    One element set is the null set - i.e. containing no elements 
	 Three element sets have only elements 
	Three element sets have two elements 
	One element set contains all three elements 
	 1+3+3+1=8
Exercise: For the set Y = {u,v,w,x} , answer the questions from the previous exercise


Power Set
Power set of a set S is the set of all subsets of S including the empty set. The cardinality of a power set of a set S of cardinality n is 2n2n. Power set is denoted as P(S)P(S).

Example −

For a set S={a,b,c,d}S={a,b,c,d} let us calculate the subsets −

Subsets with 0 elements − {∅}{∅} (the empty set)

Subsets with 1 element − {a},{b},{c},{d}{a},{b},{c},{d}
Subsets with 2 elements − {a,b},{a,c},{a,d},{b,c},{b,d},{c,d}{a,b},{a,c},{a,d},{b,c},{b,d},{c,d}
Subsets with 3 elements − {a,b,c},{a,b,d},{a,c,d},{b,c,d}{a,b,c},{a,b,d},{a,c,d},{b,c,d}
Subsets with 4 elements − {a,b,c,d}{a,b,c,d}
Hence, P(S)=P(S)=
{{∅},{a},{b},{c},{d},{a,b},{a,c},{a,d},{b,c},{b,d},{c,d},{a,b,c},{a,b,d},{a,c,d},{b,c,d},{a,b,c,d}}{{∅},{a},{b},{c},{d},{a,b},{a,c},{a,d},{b,c},{b,d},{c,d},{a,b,c},{a,b,d},{a,c,d},{b,c,d},{a,b,c,d}}
|P(S)|=24=16|P(S)|=24=16
Note − The power set of an empty set is also an empty set.

|P({∅})|=20=1|P({∅})|=20=1
\subsection{Partitioning of a Set}
Partition of a set, say S, is a collection of n disjoint subsets, say P1,P2,…PnP1,P2,…Pn that satisfies the following three conditions −

PiPi does not contain the empty set.

[Pi≠{∅} for all 0<i\leqn][Pi≠{∅} for all 0<i\leqn]
The union of the subsets must equal the entire original set.

[P1∪P2∪⋯∪Pn=S][P1∪P2∪⋯∪Pn=S]
The intersection of any two distinct sets is empty.

[Pa∩Pb={∅}, for a≠b where n≥a,b≥0][Pa∩Pb={∅}, for a≠b where n≥a,b≥0]
Example

Let S={a,b,c,d,e,f,g,h}S={a,b,c,d,e,f,g,h}
One probable partitioning is {a},{b,c,d},{e,f,g,h}{a},{b,c,d},{e,f,g,h}
Another probable partitioning is {a,b},{c,d},{e,f,g,h}{a,b},{c,d},{e,f,g,h}

%-------------------------------------------------% 
\newpage
\section*{Power Sets}
\subsection*{Worked Example}
Consider the set $Z$:
\[ Z = \{ a,b,c\}  \]
\begin{itemize}
\item[(i)] How many sets are in the power set of $Z$? 
\item[(ii)] Write out the power set of $Z$. 
\item[(iii)] How many elements are in each element set?
\end{itemize}
%----------------------------------------------%
\subsection*{Solutions to Worked Example}

\begin{itemize}


\item[(i)] There are 3 elements in $Z$. So there is $2^3 = 8$ element sets contained in the power set.

\item[(ii)] Write out the power set of $Z$.
\[ \mathcal{P}(Z) = \{ \emptyset, \{a\}, \{b\}, \{c\}, \{a,b\}, \{a,c\}, \{b,c\}, \{a,b,c\} \} \]

\item[(iii)]
\begin{itemize}
\item[$\bullet$] One element set is the null set - i.e. containing no
elements \item[$\bullet$] Three element sets have only elements \item[$\bullet$]
Three element sets have two elements \item[$\bullet$] One element set
contains all three elements \item[$\bullet$] 1+3+3+1=8
\end{itemize}
\end{itemize}
\subsection*{Exercise}
For the set $Y = \{u,v,w,x\}$ , answer the questions from the
previous exercise
\newpage

%-------------------------------------------------% 

\section*{Power Sets}
\subsection*{Worked Example}
Consider the set $Z$:
\[ Z = \{ a,b,c\}  \]
\begin{itemize}
\item[Q1] How many sets are in the power set of $Z$? 
\item[Q2] Write out the power set of $Z$. 
\item[Q3] How many elements are in each element set?
\end{itemize}
%----------------------------------------------%
\subsection*{Solutions to Worked Example}

\begin{itemize}


\item[Q1] There are 3 elements in $Z$. So there is $2^3 = 8$ element sets contained in the power set.

\item[Q2] Write out the power set of $Z$.
\[ \mathcal{P}(Z) = \{ \{0\}, \{a\}, \{b\}, \{c\}, \{a,b\}, \{a,c\}, \{b,c\}, \{a,b,c\} \]

\item[Q3]
\begin{itemize}
\item[*] One element set is the null set - i.e. containing no
elements \item[$\bullet$] Three element sets have only elements \item[$\bullet$]
Three element sets have two elements \item[$\bullet$] One element set
contains all three elements \item[$\bullet$] 1+3+3+1=8
\end{itemize}
\end{itemize}
\subsection*{Exercise}
For the set $Y = \{u,v,w,x\}$ , answer the questions from the
previous exercise


\newpage
\section*{Power Sets}
\subsection*{Worked Example}
Consider the set $Z$:
\[ Z = \{ a,b,c\}  \]
\begin{itemize}
\item[(i)] How many sets are in the power set of $Z$? 
\item[(ii)] Write out the power set of $Z$. 
\item[(iii)] How many elements are in each element set?
\end{itemize}
%----------------------------------------------%
\subsection*{Solutions to Worked Example}

\begin{itemize}


\item[(i)] There are 3 elements in $Z$. So there is $2^3 = 8$ element sets contained in the power set.

\item[(ii)] Write out the power set of $Z$.
\[ \mathcal{P}(Z) = \{ \emptyset, \{a\}, \{b\}, \{c\}, \{a,b\}, \{a,c\}, \{b,c\}, \{a,b,c\} \} \]

\item[(iii)]
\begin{itemize}
\item[$\bullet$] One element set is the null set - i.e. containing no
elements \item[$\bullet$] Three element sets have only elements \item[$\bullet$]
Three element sets have two elements \item[$\bullet$] One element set
contains all three elements \item[$\bullet$] 1+3+3+1=8
\end{itemize}
\end{itemize}
\subsection*{Exercise}
For the set $Y = \{u,v,w,x\}$ , answer the questions from the
previous exercise

%--------------------------------------------------------%
\section{Power Sets}
\begin{itemize}
\item Consider the set A where $ A = \{w,x,y,z\}$
\item There are 4 elements in set A.
\item The power set of A contains 16 element data sets.
\item \[  \mathcal{P}(A) = \{\{ x \}, \{ y \} \}  \]
\item (i.e. 1 null set, 4 single element sets, 6 two -elemnts sets, 4 three lement set and one 4- element set.)
\end{itemize}
%------------------------------------------------%


%---------------------------- %
%---------------------------------------%
\subsubsection*{Power Sets}
\begin{itemize}
\item Consider the set A where $ A = \{w,x,y,z\}$
\item There are 4 elements in set A.
\item The power set of A contains 16 element data sets.
\item \[  \mathcal{P}(A) = \{\{ x \}, \{ y \} \}  \]
\item (i.e. 1 null set, 4 single element sets, 6 two -elemnts sets, 4 three lement set and one 4- element set.)
\end{itemize}

%------------------------------------------------------%
\end{document}
