
\section{Elements of a Set}

A set is defined completely by its elements. Formally, sets X and Y are the same set if they have the same elements; that is, if every element of X is also an element of Y, and vice versa. For example, suppose we define:

\[ F = \{x | \mbox{(x is an integer)} \mbox{ and } 0 < x < 4)\}  \]
%Then F=A. The definition also makes it clear that {1,2,3} = (3,2,1} = {1,1,2,3,2}. Note also that if X and Y are both empty, then they are equal, justifying referring to "the" empty set.

The equivalence of empty sets has a \textbf{\textit{metaphysical}} consequence for some theories of the metaphysics of properties that define the property of being x as simply the set of all x, then if the two properties are uninstantiated or coextensive they are equivalent - under this theory, because there are no unicorns and there are no pixies, the property of being a unicorn and being a pixie are the same - but if there were a unicorns and pixies, we could tell them apart. (See Universals for more on this.)

\textbf{Elements and subsets}

The $\in$  sign indicates set membership. If x is an element (or "member") of a set X, we write $ \in X$; e.g. $3\in A$. (We may also say ``X contains x" and ``A contains 3")

A very important notion is that of a subset. X is a subset of Y, written $X \subseteq Y$ (sometimes simply as$X \subset Y$), if every element of X is also an element of Y. From before $C\subseteq A \subseteq E$.


\textbf{Sets containing Sets}


Sets can of course be elements of other sets; for example we can form the set $G = \{A,B,C,D,E\}$, whose five elements are the sets we considered earlier. Then, for instance, $A\in G$. (Note that this is very different from saying $A\subseteq G$)

%The tilde (~) is as usual used for negation; e.g. $\tilde(A\in B)$.
