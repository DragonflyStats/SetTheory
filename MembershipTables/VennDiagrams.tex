\documentclass{beamer}

\usepackage{graphics}
\usepackage{framed}
\usepackage{amsmath}
\begin{document}

\begin{frame}



\begin{center}
{\Huge Venn Diagrams  } \\

\bigskip{\Large kobriendublin.wordpress.com}\\ \vspace{0.4cm}
{\Large Twitter: @statslabdublin}
\end{center}


\end{frame}

\begin{frame}
\Large
\begin{itemize}
\item A college teaches courses in the following subjects areas: mathematics, computing and statistics.
\item Students in the college may choose their courses from these three subject areas.
\item Students are not obliged to take courses from these three subject areas, and may instead take courses in other subject areas. 
\end{itemize}

 
\end{frame}
\begin{frame}
\Large
\begin{itemize}
\item  Let the subject areas be represented by the letters \textbf{M} for mathematics, \textbf{C} for computing and \textbf{S} for statistics. \item Draw a labelled Venn diagram showing the areas \textbf{M}, \textbf{C}, and \textbf{S} in such a way as to represent the students studying at the college. \item On your diagram show the number of students studying in each region of the Venn diagram.
\end{itemize}
\end{frame}


\begin{frame}
\Large
\begin{itemize}
\item Currently 600
students are enrolled in the college. 
\item 300 students are taking mathematics courses.
\item 120 student are taking statistics courses.
\item 380 students are taking computing courses. 
\end{itemize}
\end{frame}
\begin{frame}
\Large
\begin{itemize}
\item 40 students study courses from all three subject
areas. 
\item 200 mathematics students are taking computing courses as well. \item 60 computing students
are also takings statistics courses. \item  70 statistics students are also taking mathematics course.
\end{itemize}
\end{frame}
\begin{frame}

\frametitle{Questions}
\Large
\begin{itemize}
\item[(i)] How many students study none of these courses at all?
\item[(ii)] How many students are taking mathematics courses but not computing or statistics courses.
\item[(iii)] How many students study courses from precisely two of these subject
areas?

\end{itemize}

\end{frame}
\end{document}
