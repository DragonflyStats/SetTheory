
\subsection{Types of Sets}
Sets can be classified into many types. Some of which are finite, infinite, subset, universal, proper, singleton set, etc.

\subsection{Finite Set}
A set which contains a definite number of elements is called a finite set.

Example − \[S={x|x∈NS={x|x∈N and 70>x>50}70>x>50}\]
\subsection{Infinite Set}
A set which contains infinite number of elements is called an infinite set.

Example − S={x|x∈NS={x|x∈N and x>10}x>10}
\subsection{Subset}
A set X is a subset of set Y (Written as X⊆YX⊆Y) if every element of X is an element of set Y.

Example 1 − Let, X={1,2,3,4,5,6}X={1,2,3,4,5,6} and Y={1,2}Y={1,2}. Here set Y is a subset of set X as all the elements of set Y is in set X. Hence, we can write Y⊆XY⊆X.

Example 2 − Let, X={1,2,3}X={1,2,3} and Y={1,2,3}Y={1,2,3}. Here set Y is a subset (Not a proper subset) of set X as all the elements of set Y is in set X. Hence, we can write Y⊆XY⊆X.

\subsection{Proper Subset}
The term “proper subset” can be defined as “subset of but not equal to”. A Set X is a proper subset of set Y (Written as X⊂YX⊂Y) if every element of X is an element of set Y and |X|<|Y||X|<|Y|.

Example − Let, X={1,2,3,4,5,6}X={1,2,3,4,5,6} and Y={1,2}Y={1,2}. Here set Y⊂XY⊂X since all elements in YY are contained in XX too and XX has at least one element is more than set YY.

\subsection{Universal Set}
It is a collection of all elements in a particular context or application. All the sets in that context or application are essentially subsets of this universal set. Universal sets are represented as UU.

Example − We may define UU as the set of all animals on earth. In this case, set of all mammals is a subset of UU, set of all fishes is a subset of UU, set of all insects is a subset of UU, and so on.

\subsection{Empty Set or Null Set}
An empty set contains no elements. It is denoted by ∅∅. As the number of elements in an empty set is finite, empty set is a finite set. The cardinality of empty set or null set is zero.

Example − \[S={x|x∈NS={x|x∈N and 7<x<8}=∅7<x<8}=∅\]

\frametitle{The Empty Set}

Another set which has a special letter to denote it is the set containing no elements. This is called
the empty or null set and denoted by the symbol (D. _
Example 2.6 The set of integers m such that m2 = 5 is the empty set. We could write
{mEZ:m2=5}:@.
\smallskip 
%----------------------------------------- %




\subsection{Singleton Set or Unit Set}
Singleton set or unit set contains only one element. A singleton set is denoted by {s}{s}.

Example − S={x|x∈N, 7<x<9}S={x|x∈N, 7<x<9} = {8}{8}
\subsection{Equal Set}
If two sets contain the same elements they are said to be equal.

Example − If A={1,2,6}A={1,2,6} and B={6,1,2}B={6,1,2}, they are equal as every element of set A is an element of set B and every element of set B is an element of set A.

\section{Equivalent Set}
If the cardinalities of two sets are same, they are called equivalent sets.

Example − If A={1,2,6}A={1,2,6} and B={16,17,22}B={16,17,22}, they are equivalent as cardinality of A is equal to the cardinality of B. i.e. |A|=|B|=3|A|=|B|=3
\subsection{Overlapping Set}
Two sets that have at least one common element are called overlapping sets.

In case of overlapping sets −

n(A∪B)=n(A)+n(B)−n(A∩B)n(A∪B)=n(A)+n(B)−n(A∩B)
n(A∪B)=n(A−B)+n(B−A)+n(A∩B)n(A∪B)=n(A−B)+n(B−A)+n(A∩B)
n(A)=n(A−B)+n(A∩B)n(A)=n(A−B)+n(A∩B)
n(B)=n(B−A)+n(A∩B)n(B)=n(B−A)+n(A∩B)
Example − Let, A={1,2,6}A={1,2,6} and B={6,12,42}B={6,12,42}. There is a common element ‘6’, hence these sets are overlapping sets.

\section{Disjoint Set}
Two sets A and B are called disjoint sets if they do not have even one element in common. Therefore, disjoint sets have the following properties −

n(A∩B)=∅n(A∩B)=∅
n(A∪B)=n(A)+n(B)n(A∪B)=n(A)+n(B)
Example − Let, A={1,2,6}A={1,2,6} and B={7,9,14}B={7,9,14}, there is not a single common element, hence these sets are overlapping sets.


%------------------------------------------------------%

\section*{Complement of a Set}
%(2.3.1) 
Consider the universal set $U$ such that
\[U=\{2,4,6,8,10,12,15\} \]
For each of the sets $A$,$B$,$C$ and $D$, specify the complement sets.
{
	\LARGE
\begin{center}
\begin{tabular}{|c|c|}
  \hline
Set &\phantom{sp} Complement \phantom{sp}\\
\hline \phantom{sp} $A=\{4,6,12,15\}$ \phantom{sp} &
$A^{\prime}=\{2,8,10\}$ \\ \hline $B=\{4,8,10,15\}$ & \\ \hline
$C=\{2,6,12,15\}$ & \\ \hline $D=\{8,10,15\}$ & \\ \hline

\end{tabular}
\end{center}
}



{Equal and Equivalent Sets}

%%- \vspace{-1cm}
Difference between equal sets and equivalent sets

\begin{itemize}
\item Consider the sets \textbf{A} and \textbf{B}
\[ \boldsymbol{A} = \{ 1,2,3,4,5,6 \} \] 
\[ \boldsymbol{B} = \{1,2,3,4,5,6 \} \]
%%- \vspace{0.2cm}
\item \textbf{A} and \textbf{B} are \textit{\textbf{equal}} sets because \textit{\textbf{all}} their
elements are precisely the \textit{\textbf{same}}.
\end{itemize}



%---------------------------------------------------%


{Equal and Equivalent Sets}

%%- \vspace{-0.7cm}
Difference between equal sets and equivalent sets

\begin{itemize}
\item Consider the sets \textbf{C} and \textbf{D}
\[ \boldsymbol{C} = \{a,b,c,d,e,f\} \]  \[ \boldsymbol{D} = \{3,4,5,6,7,8\} \]
%%- \vspace{0.2cm}
\item \textbf{C} and \textbf{D} are \textit{\textbf{equivalent}} sets
because the cardinality of both the sets is the same (i.e. 6.)
\item However \textbf{C} and \textbf{D} are not equal, as they are comprised of different elements.
\end{itemize}






%---------------------------------------------------%


{Equal and Equivalent Sets}

%%- \vspace{-1cm}

\begin{itemize}
\item Necessarily all equal sets are equivalent sets.
\item But are equivalent sets equal sets?

\item No, because equivalent sets are sets that have the \textit{\textbf{same}} cardinality but equal sets are sets that all
their elements are precisely the \textit{\textbf{same}}. 
\end{itemize}

\textbf{Example}:
\begin{itemize} \item \textbf{X}=\{p,q,r\} and \textbf{Y}=\{1,2,3\} are equivalent sets 
\item \textbf{E} =\{m,n,o,p\}
and \textbf{F}=\{m,n,o,p\} are equal and equivalent sets
\end{itemize}
\end{document}
